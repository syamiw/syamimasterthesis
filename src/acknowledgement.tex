\chapter*{謝辞}
\addcontentsline{toc}{chapter}{謝辞}
\label{thanks}

本研究を進めるにあたり、ご指導いただいた慶應義塾大学教授村井純博士、同大学環境情報学部教授中村修博士、同学部教授楠本博之博士、同学部教授高汐一紀博士、同学部教授Rodney D.Van Meter III 博士、同学部教授植原啓介博士、同学部教授三次仁博士、同学部教授中澤仁博士、同学部教授手塚悟博士、同学部教授武田圭史博士、同学部教授大越匡博士、同大学政策・メディア研究科特任准教授佐藤雅明博士、同大学政策・メディア研究科特任教授鈴木茂哉博士、同大学SFC 研究所上席所員兼早稲田大学商学学術院大学院経営管理研究科斉藤賢爾博士に感謝いたします。特に斉藤氏には重ねて感謝致します。研究活動を通して技術的視点、社会学的視点等の様々な視点から私の研究活動に対して助言を頂き、私だけではなし得なかった深い思考と知見を頂き、貴重な学びを経験させて頂くことが出来ました。これらの経験は私の人生において一人の人として、また学ぶ者として、大切で素敵な財産として残りました。博士の指導なしには、本修士論文を執筆することは出来ませんでした。

村井・中村・楠本・高汐・バンミーター・植原・三次・中澤・手塚・武田・大越 合同研究プロジェクトに所属している学部生、大学院生、卒業生の皆さまに感謝いたします。研究会に所属する多くの方々が各々の分野・研究で奮闘している姿を見て学んだことで、私の研究生活はより充実したものとなりました。異なる分野同士が触れ合い、学び合う環境に出会えたことをとても嬉しく感じます。

また、NECO 研究グループとして多くの意見・発想・知見を与えてくださった、島津翔太氏、江頭叙那氏、金城奈菜海氏、長田琉羽里氏、井出匡紀氏、田崎和輝氏、松藤舜氏、月館力哉氏、須永陸也氏、梅沢康生氏、祖父江圭悟氏、元田遼太氏、森優貴氏、山本朋義氏、野口心音氏、林沙有良氏、霜田哲之介氏、渡辺孝亮氏、吉崎航輝氏に感謝致します。皆様には、私の研究に対する多くの助言や発想を頂いただけでなく、研究活動における多くの学びを経験させて頂きました。多くの出会いと学びの環境であるSFC に感謝致します。多様な学問領域に触れ、学生同士で議論し思考することが出来ました。幸せで素敵な時間でした。

最後に、これまで私を育て、見守り、学びの機会並びに本修士論文執筆の機会を与えて
頂いた、父泰生、母尚子、弟友賀に感謝致します。



%%% Local Variables:
%%% mode: japanese-latex
%%% TeX-master: "../yummy_bthesis"
%%% End:
