\chapter{評価}
\label{discussion}
%結果を解釈・評価する
本章では、\ref{experiment}章で行った実装のの評価について述べる。

\section{評価内容}
評価として、以下の内容を検証する。実装が実際に利用できるという観点から、1つ目は類似度の計算が一般的な利用に問題ない時間であるかを測定する。2つ目は、実際に利用をしてもらい、読解時間を短縮することができるかについて検証を行う。

\section{評価のための利用規約}
提案手法の評価のために、利用規約をAlpha社、Beta社、Gamma社、Delta社、Epsilon社、Zeta社、Eta社の8つ用意した。これらは、実在する会社やサービスの利用規約を実験用に会社名やサービス名を置き換えたり、一部条項の書き換えをおこなったものである。また、評価で利用規約の読解に要した時間を測定するために、条文の数を揃える作業を行なっている。利用規約については、以下のような観点から選んだ。
\begin{itemize}
  \item 条文数が極端に多くない
  \item 可能な限り違う種類のサービスを展開している
\end{itemize}

\begin{table}[h]
  \centering
  \caption{実験用利用規約の一覧}
  \begin{tabular}{cc}
  \hline
  実験用社名    & 主に元にした利用規約\\ \hline\hline
  Alpha社   & 掲示版・SNS系向きの利用規約の雛形(ひな型)\footnote{https://kiyaku.jp/hinagata/sns.html}\\ \hline\
  Beta社    & Annict\footnote{https://annict.com/terms}\\ \hline\
  Gamma社   & 鎌倉新書\footnote{https://www.kamakura-net.co.jp/servicepolicy/}\\ \hline\
  Delta社   & クックパッド\footnote{https://cookpad.com/terms/free}\\ \hline\
  Epsilon社 & 第一法規\footnote{https://www.daiichihoki.co.jp/support/rules/}\\ \hline\
  Zeta社    & IRIAM\footnote{https://www.live.iriam.com/terms}\\ \hline\
  Eta社     & 三越伊勢丹WEB会員規約\footnote{https://www.mistore.jp/shopping/help/guide/terms\_h.html}\\ \hline\
  Theta社   & Z会ソリューションズ\footnote{https://www.zkai.co.jp/assess/terms}\\ \hline
  \end{tabular}
\end{table}

\section{評価のためのクイズ}
本提案手法により利用者の理解度が下がっていないかを確認するために、クイズを設定した。クイズに内容は、評価のための利用規約の中から広く選び、また、一部書き換えた条文などをクイズとして出題をした。具体的なクイズの内容は付録に記す。

%%% Local Variables:
%%% mode: japanese-latex
%%% TeX-master: "./thesis"
%%% End:
