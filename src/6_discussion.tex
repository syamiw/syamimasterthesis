\chapter{評価}
\label{discussion}
%結果を解釈・評価する
本章では、\ref{experiment}章で行った実装のの評価について述べる。

\section{評価内容}
評価として、以下の内容を検証する。実装が実際に利用できるという観点から、1つ目は類似度の計算が一般的な利用に問題ない時間であるかを測定する。2つ目は、実際に利用をしてもらい、読解時間を短縮することができるかについて検証を行う。

\section{評価のための利用規約}
\label{sec:評価のための利用規約}
提案手法の評価のために、利用規約をAlpha社、Beta社、Gamma社、Delta社、Epsilon社、Zeta社、Eta社の8つ用意した。これらは、実在する会社やサービスの利用規約を実験用に会社名やサービス名を置き換えたり、一部条項の書き換えをおこなったものである。また、評価で利用規約の読解に要した時間を測定するために、条文の数を揃える作業を行なっている。利用規約については、以下のような観点から選んだ。
\begin{itemize}
  \item 条文数が極端に多くない
  \item 可能な限り違う種類のサービスを展開している
  \item 提案システムを通して見た際に判別される部分がある程度一定になる
\end{itemize}
これらの条件から、実在する利用規約の選定を行い、これらについて社名の変更などや、提案システムの判別が一定に起こるように、条文の編集などを行なった。

\begin{table}[h]
  \centering
  \caption{実験用利用規約の一覧}
  \begin{tabular}{cc}
  \hline
  実験用社名    & 主に元にした利用規約\\ \hline\hline
  Alpha社   & 掲示版・SNS系向きの利用規約の雛形(ひな型)\tablefootnote{https://kiyaku.jp/hinagata/sns.html}\\ \hline\
  Beta社    & Annict\tablefootnote{https://annict.com/terms}\\ \hline\
  Gamma社   & 鎌倉新書\tablefootnote{https://www.kamakura-net.co.jp/servicepolicy/}\\ \hline\
  Delta社   & クックパッド\tablefootnote{https://cookpad.com/terms/free}\\ \hline\
  Epsilon社 & 第一法規\tablefootnote{https://www.daiichihoki.co.jp/support/rules/}\\ \hline\
  Zeta社    & IRIAM\tablefootnote{https://www.live.iriam.com/terms}\\ \hline\
  Eta社     & 三越伊勢丹WEB会員規約\tablefootnote{https://www.mistore.jp/shopping/help/guide/terms\_h.html}\\ \hline\
  Theta社   & Z会ソリューションズ\tablefootnote{https://www.zkai.co.jp/assess/terms}\\ \hline
  \end{tabular}
\end{table}
評価の際はこれらの実験用社名については被験者にわかるように明示をしている。

\section{評価のためのクイズ}
\label{sec:評価のためのクイズ}
本提案手法により通常の利用規約を読んだときよりも著しく利用者の理解度が下がっていると、本提案手法は読解の阻害になってしまうと考えられるため、そのようなことが起きていないかを確認するために、クイズを設定した。クイズに内容は、評価のための利用規約の中から広く選び、また、一部書き換えた条文などをクイズとして出題をした。問題数は各利用規約について2択のクイズを2問となっている。1問であると利用規約間の難易度の調整が困難であるため、2問として難易度の平易化を行なっている。具体的なクイズの内容は付録に記す。

\section{評価用実験システムの環境}
本提案システムが有効に機能するかを実験するために、評価用のシステムを構築した。容易に実験することが可能なように、PythonのフレームワークであるFlaskでWebサイトとして提案システムが容易に扱えるようにした。そのシステムをAzure Web App Service上に展開をし、そこへのアクセスをもって実験を行えるようにした。

一般的なインターネットユーザーを想定し、かつ、利用規約を読んだことがあるという観点から、クラウドソーシングサービスである、クラウドワークス\footnote{https://crowdworks.jp/}を用いて被験者を募った。クラウドワークスの利用者は、登録時にクラウドワークスの利用規約を読んだことがあるので、利用規約を読まなければいけないという経験がある被験者の集団を作ることができる。なお、報酬は、1人あたり100円とした。被験者には、初めにページにて本提案手法の趣旨の説明を行なった。
\begin{figure}[h]
  \begin{center}
      \includegraphics[width=16cm]{img/teststart1.png}
      \caption{実験概要説明ページ}
      \label{img:実験概要説明ページ}
  \end{center}
\end{figure}
被験者への説明は、「利用規約を読みやすくするためのツールのテスト」であるという表現にとどめ、本来の目的である、利用規約の読解時間短縮ということは伝えていない。これは、本来の実験の目的を最初に伝えてしまうと、被験者の読解時間の変動が起きてしまうと考えられるからである。

被験者は説明を読んだ後、利用規約を8つ読んでいく。提案システムとシステムを通さない通常の利用規約表示を比較するため、順番はランダムでかつ4つずつを読む形にした。8つの利用規約の順番もランダムであり、ランダムでどのような順序で実験が行われるかは予め被験者に通知をした。利用規約を読む時間をそれぞれ環境を作成した。
\begin{figure}[h]
  \begin{center}
      \includegraphics[width=16cm]{img/teststart2.png}
      \caption{実験順序説明ページ}
      \label{img:実験順序説明ページ}
  \end{center}
\end{figure}

そのあとは利用規約を表示し、「同意する」「同意しない」をクリックする。その後、\ref{sec:評価のためのクイズ}節で示したクイズが表示される。クイズに回答した後は次の利用規約を表示する表示が繰り返され、全ての利用規約を閲覧し終わったあとはアンケートに回答を依頼した。

\subsection{結果}


%%% Local Variables:
%%% mode: japanese-latex
%%% TeX-master: "./thesis"
%%% End:
