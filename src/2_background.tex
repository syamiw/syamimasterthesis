\chapter{背景}
\label{background}
本章では本研究の背景について述べる。

\section{利用規約}
本節では、利用規約について法的な立場の整理を行い、その重要性について論じる。

\subsection{法的根拠}
「利用規約」という用語は法令用語ではなく、法的には約款の一種であると考えられる\cite{itakura2013}。2020年4月1日以前の民法には利用規約に関する規定は設定されていなかった。当時は当事者間において約款により個別に契約が結ばれていると解釈をなされていたが、これらは事業者が一方的に作成したものであり、利用者が条項の内容を認識していないということが多い。本来、両者の認識のもと合意に基づいて契約がなされるという前提により法的拘束力があるとみなされるべきである。しかし、実態として条項の内容を認識していない利用者に個別の契約交渉をさせるということは困難であるが、一方で定型的な内容が想定される契約類型においては約款の法的拘束力を認めないと、円滑な取引を阻害させることになる。また、約款に含まれる条項が契約内容になることが争われた裁判では個別のケースごとに判断が分かれるなど透明性にも課題があった\cite{hashimoto2021}。

これらの要請をもとに、2020年4月1日に民法の改正が行われた\cite{moj2020minpo}。これにより、利用規約のような不特定多数と契約を執り行うような約款を「定型約款」として定義された。定型約款として

%\begin{screen}
%  (定型約款の合意)\\
%  第五百四十八条の二\\
%  定型取引(ある特定の者が不特定多数の者を相手方として行う取引であって、その内容の全部又は一部が画一的であることがその双方にとって合理的なものをいう。以下同じ。)を行うことの合意(次条において「定型取引合意」という。)をした者は、次に掲げる場合には、定型約款(定型取引において、契約の内容とすることを目的としてその特定の者により準備された条項の総体をいう。以下同じ。)の個別の条項についても合意をしたものとみなす。
%  一 定型約款を契約の内容とする旨の合意をしたとき。
%  二 定型約款を準備した者(以下「定型約款準備者」という。)があらかじめその定型約款を契約の内容とする旨を相手方に表示していたとき。
%  2 前項の規定にかかわらず、同項の条項のうち、相手方の権利を制限し、又は相手方の義務を加重する条項であって、その定型取引の態様及びその実情並びに取引上の社会通念に照らして第一条第二項に規定する基本原則に反して相手方の利益を一方的に害すると認められるものについては、合意をしなかったものとみなす。
%\end{screen}


%\if0
%\begin{figure}[h]
%    \begin{center}
%        \includegraphics[scale=0.4]{./img/hashrate.png}
%        \caption{2017年1月のハッシュレート分布 出典:Blockchain.info\cite{bitcoinhashrate}}
%        \label{img:hashrate}
%    \end{center}
%\end{figure}
%\fi