\chapter{結論}
\label{conclusion}
%まとめる、次に続く研究への課題を書く、自分のメッセージを書く

本章では,本研究のまとめと今後の課題を示す.

\section{本研究のまとめ}

\section{本研究の課題}
本節では、本研究で提案したシステムの課題と展望について述べる。

\subsection{構想}
本研究の提案をより、社会実装しやすくするためのものとして、利用規約図書館のような構想がある。この構想は筆者の卒業論文で提案されている利用規約をユーザー同士でとりまとめて保存しておくシステムである。このような仕組みを構築することで、同じ約款を大量の人が同意する点に着目し、以下のようなメリットが考えられる。
\begin{enumerate}
  \item 利用規約の変更などを利用者同士で確認し、追従することができる。
  \item 本研究の実装の自然言語処理などの手順を誰か1人が取ることで、その成果を他の利用者も利用することができる。
\end{enumerate}
本研究では特に、文ベクトルの生成に利用者の計算リソースを消費してしまうという課題があり、さらに、文量が非常に多い約款の場合はシステムの利用のための時間がかかってしまうため、この仕組みをもとにシステムを稼働することができれば、より利用しやすいシステムになると考えられる。しかし、この提案にも維持者や管理人が必要になるために、このコストを誰が負担するかという問題がある。論文では、この問題を克服するために、適格消費者団体のような、現在利用規約の監視や裁判を消費者側の立場で行なっている団体が管理することを提案している。 


%%% Local Variables:
%%% mode: japanese-latex
%%% TeX-master: "../thesis"
%%% End:
