\appendix
\chapter{実験に利用した利用規約}
実験に利用した利用規約を以下に示す。なお、利用した利用規約は実在する利用規約をベースに社名やサービス名などを置き換え、実験用に条項の一部を変更したものである。

\section{Alpha社}
第1条(適用)

    本規約は,ユーザーと当社との間の本サービスの利用に関わる一切の関係に適用されるものとします。

    当社は本サービスに関し,本規約のほか,ご利用にあたってのルール等,各種の定め(以下,「個別規定」といいます。)をすることがあります。これら個別規定はその名称のいかんに関わらず,本規約の一部を構成するものとします。

    本規約の規定が前条の個別規定の規定と矛盾する場合には,個別規定において特段の定めなき限り,個別規定の規定が優先されるものとします。

第2条(利用登録)

    本サービスにおいては,登録希望者が本規約に同意の上,当社の定める方法によって利用登録を申請し,当社がこの承認を登録希望者に通知することによって,利用登録が完了するものとします。

    当社は,利用登録の申請者に以下の事由があると判断した場合,利用登録の申請を承認しないことがあり,その理由については一切の開示義務を負わないものとします。 

        利用登録の申請に際して虚偽の事項を届け出た場合

        本規約に違反したことがある者からの申請である場合

        反社会的勢力に該当したとき、または反社会的勢力と関係を有している者からの申請である場合

        その他,当社が利用登録を相当でないと判断した場合

第3条(ユーザーIDおよびパスワードの管理)

    ユーザーは,自己の責任において,本サービスのユーザーIDおよびパスワードを適切に管理するものとします。

    ユーザーは,いかなる場合にも,ユーザーIDおよびパスワードを第三者に譲渡または貸与し,もしくは第三者と共用することはできません。当社は,ユーザーIDとパスワードの組み合わせが登録情報と一致してログインされた場合には,そのユーザーIDを登録しているユーザー自身による利用とみなします。

    ユーザーID及びパスワードが第三者によって使用されたことによって生じた損害は,当社に故意又は重大な過失がある場合を除き,当社は一切の責任を負わないものとします。

第4条(利用料金および支払方法)

    ユーザーは,本サービスの有料部分の対価として,当社が別途定め,本ウェブサイトに表示する利用料金を,当社が指定する方法により支払うものとします。

    ユーザーは、ユーザーの都合、または携帯電話会社の都合、その他当社の責に帰するべき事由により本サービスまたは本コンテンツ等を利用できなかった場合においても情報料を支払うものとします。

    当社は、いかなる理由によっても、いったんお支払いいただいた情報料を一切返還いたしません。

    情報料は、本契約締結日の属する月(締結日を問わず満1ヵ月として計算します)から本契約が解除された日の属する月(解除日を問わず満1ヵ月として計算します)まで毎月お支払いいただきます。

    本契約締結日の属する月に本契約が解除された場合は1ヵ月分の情報料をお支払いいただきます。

    本サービスのご利用には情報料の他に別途通信料がかかります。

    ユーザーがパケットサービスをご利用の場合には送受信の通信料がかかります。

    ユーザーが利用料金の支払を遅滞した場合には,ユーザーは年14.6%の割合による遅延損害金を支払うものとします。

    本コンテンツの利用料は無料ですが一部有料コンテンツを含みます。また、事前告知をした上で無料コンテンツの一部を有料コンテンツに変更することがあります。

    利用者の要請・事情等によるサービスは当社規定の方法による見積もりといたします。 

    当社は、利用者への事前通知によって本サービスの基本利用料金を改定することができるものとします。 

第5条(禁止事項)

ユーザーは,本サービスの利用にあたり,以下の行為をしてはなりません。

    法令または公序良俗に違反する行為

    犯罪行為に関連する行為

    当社,本サービスの他のユーザー,または第三者のサーバーまたはネットワークの機能を破壊したり,妨害したりする行為

    当社のサービスの運営を妨害するおそれのある行為

    他のユーザーに関する個人情報等を収集または蓄積する行為

    不正アクセスをし,またはこれを試みる行為

    他のユーザーに成りすます行為

    当社のサービスに関連して,反社会的勢力に対して直接または間接に利益を供与する行為

    当社,本サービスの他のユーザーまたは第三者の知的財産権,肖像権,プライバシー,名誉その他の権利または利益を侵害する行為

    以下の表現を含み,または含むと当社が判断する内容を本サービス上に投稿し,または送信する行為 

        過度に暴力的な表現

        露骨な性的表現

        人種,国籍,信条,性別,社会的身分,門地等による差別につながる表現

        自殺,自傷行為,薬物乱用を誘引または助長する表現

        その他反社会的な内容を含み他人に不快感を与える表現

    以下を目的とし,または目的とすると当社が判断する行為

        営業,宣伝,広告,勧誘,その他営利を目的とする行為(当社の認めたものを除きます。)

        性行為やわいせつな行為を目的とする行為

        面識のない異性との出会いや交際を目的とする行為

        他のユーザーに対する嫌がらせや誹謗中傷を目的とする行為

        当社,本サービスの他のユーザー,または第三者に不利益,損害または不快感を与えることを目的とする行為

        その他本サービスが予定している利用目的と異なる目的で本サービスを利用する行為

    宗教活動または宗教団体への勧誘行為

    その他,当社が不適切と判断する行為

第6条(本サービスの提供の停止等)

    当社は,以下のいずれかの事由があると判断した場合,ユーザーに事前に通知することなく本サービスの全部または一部の提供を停止または中断することができるものとします。

        本サービスにかかるコンピュータシステムの保守点検または更新を行う場合

        地震,落雷,火災,停電または天災などの不可抗力により,本サービスの提供が困難となった場合

        コンピュータまたは通信回線等が事故により停止した場合

        その他,当社が本サービスの提供が困難と判断した場合

    当社は,本サービスの提供の停止または中断により,ユーザーまたは第三者が被ったいかなる不利益または損害についても,一切の責任を負わないものとします。

第7条(著作権)

    ユーザーは,自ら著作権等の必要な知的財産権を有するか,または必要な権利者の許諾を得た文章,画像や映像等の情報に関してのみ,本サービスを利用し,投稿ないしアップロードすることができるものとします。

    ユーザーが本サービスを利用して投稿ないしアップロードした文章,画像,映像等の著作権については,当社に移転されるものとします。ただし,著作者人格権についてはこれに含まないものとします。

    前項本文の定めるものを除き,本サービスおよび本サービスに関連する一切の情報についての著作権およびその他の知的財産権はすべて当社または当社にその利用を許諾した権利者に帰属し,ユーザーは無断で複製,譲渡,貸与,翻訳,改変,転載,公衆送信(送信可能化を含みます。),伝送,配布,出版,営業使用等をしてはならないものとします。

第8条(利用制限および登録抹消)

    当社は,ユーザーが以下のいずれかに該当する場合には,事前の通知なく,投稿データを削除し,ユーザーに対して本サービスの全部もしくは一部の利用を制限しまたはユーザーとしての登録を抹消することができるものとします。

        本規約のいずれかの条項に違反した場合

        登録事項に虚偽の事実があることが判明した場合

        決済手段として当該ユーザーが届け出たクレジットカードが利用停止となった場合

        料金等の支払債務の不履行があった場合

        当社からの連絡に対し,一定期間返答がない場合

        本サービスについて,最終の利用から一定期間利用がない場合

        その他,当社が本サービスの利用を適当でないと判断した場合

    前項各号のいずれかに該当した場合,ユーザーは,当然に当社に対する一切の債務について期限の利益を失い,その時点において負担する一切の債務を直ちに一括して弁済しなければなりません。

    当社は,本条に基づき当社が行った行為によりユーザーに生じた損害について,一切の責任を負いません。

第9条(退会)

    ユーザーは,当社の定める退会手続により,本サービスから退会できるものとします。

第10条(保証の否認および免責事項)

    当社は,本サービスに事実上または法律上の瑕疵(安全性,信頼性,正確性,完全性,有効性,特定の目的への適合性,セキュリティなどに関する欠陥,エラーやバグ,権利侵害などを含みます。)がないことを明示的にも黙示的にも保証しておりません。

    当社は,本サービスに起因してユーザーに生じたあらゆる損害について、当社の故意又は重過失による場合を除き、一切の責任を負いません。ただし,本サービスに関する当社とユーザーとの間の契約(本規約を含みます。)が消費者契約法に定める消費者契約となる場合,この免責規定は適用されません。

    前項ただし書に定める場合であっても,当社は,当社の過失(重過失を除きます。)による債務不履行または不法行為によりユーザーに生じた損害のうち特別な事情から生じた損害(当社またはユーザーが損害発生につき予見し,または予見し得た場合を含みます。)について一切の責任を負いません。また,当社の過失(重過失を除きます。)による債務不履行または不法行為によりユーザーに生じた損害の賠償は,ユーザーから当該損害が発生した月に受領した利用料の額を上限とします。

    当社は,本サービスに関して,ユーザーと他のユーザーまたは第三者との間において生じた取引,連絡または紛争等について一切責任を負いません。

第11条(サービス内容の変更等)

    当社は,ユーザーへの事前の告知をもって、本サービスの内容を変更、追加または廃止することがあり、ユーザーはこれを承諾するものとします。

第12条(利用規約の変更)

    当社は以下の場合には、ユーザーの個別の同意を要せず、本規約を変更することができるものとします。

        本規約の変更がユーザーの一般の利益に適合するとき。

        本規約の変更が本サービス利用契約の目的に反せず、かつ、変更の必要性、変更後の内容の相当性その他の変更に係る事情に照らして合理的なものであるとき。

    当社はユーザーに対し、前項による本規約の変更にあたり、事前に、本規約を変更する旨及び変更後の本規約の内容並びにその効力発生時期を通知します。

第13条(個人情報の取扱い)

    当社は,本サービスの利用によって取得する個人情報については,当社「プライバシーポリシー」に従い適切に取り扱うものとします。

第14条(通知または連絡)

    ユーザーと当社との間の通知または連絡は、電話によって行うものとします。

    当社は、ユーザーから、当社が別途定める方式に従った変更届け出がない限り、現在登録されている連絡先が有効なものとみなして当該連絡先へ通知または連絡を行い、これらは、発信時にユーザーへ到達したものとみなします。

    当社は、本サービスを常に良好な状態でご利用いただくために、システムの定期保守を行います。この場合、当社所定のウェブサイト等において定期保守の予定を告知します。

第15条(権利義務の譲渡の禁止)

    ユーザーは,当社の書面による事前の承諾なく,利用契約上の地位または本規約に基づく権利もしくは義務を第三者に譲渡し,または担保に供することはできません。

第16条(準拠法・裁判管轄)

    本規約の解釈にあたっては,日本法を準拠法とします。

    本サービスに関して紛争が生じた場合には,当社の本店所在地を管轄する裁判所を専属的合意管轄とします。

\section{Beta社}
第1条 (適用)

    本規約は、本サービスの提供条件及び本サービスの利用に関する運営者と登録ユーザーとの間の権利義務関係を定めることを目的とし、登録ユーザーと運営者との間の本サービスの利用に関わる一切の関係に適用されます。

    本規約の内容と、その他本規約以外の箇所に表示された本サービスの説明等とが異なる場合は、本規約の規定が優先して適用されるものとします。

第2条 (定義)

本規約において使用する以下の用語は、それぞれ以下に定める意味を有するものとします。

    「サービス利用契約」とは、本規約及び運営者と登録ユーザーの間で締結する、本サービスの利用契約を意味します。

    「知的財産権」とは、著作権、特許権、実用新案権、意匠権、商標権その他の知的財産権 (それらの権利を取得し、またはそれらの権利につき登録等を出願する権利を含みます。) を意味します。

    「投稿データ」とは、登録ユーザーが本サービスを利用して投稿その他送信するコンテンツ (文章、画像、動画その他のデータを含みますがこれらに限りません。) を意味します。

    「登録ユーザー」とは、第3条 (登録) に基づいて本サービスの利用者として登録された個人または法人を意味します。

    「本サービス」とは、運営者が提供する「Beta」という名称のサービス (理由の如何を問わずサービスの名称または内容が変更された場合は、当該変更後のサービスを含みます。) を意味します。

第3条 (登録)

    本サービスの利用を希望する者 (以下「登録希望者」と言います。) は、本規約を遵守することに同意し、かつ運営者の定める一定の情報 (以下「登録事項」と言います。) を運営者の定める方法で運営者に提供することにより、運営者に対し、本サービスの利用の登録を申請することができます。

    運営者は、運営者の基準に従って、第1項に基づいて登録申請を行った登録希望者 (以下「登録申請者」と言います。) の登録の可否を判断し、運営者が登録を認める場合にはその旨を登録申請者に通知します。登録申請者の登録ユーザーとしての登録は、運営者が本項の通知を行ったことをもって完了したものとします。

    前項に定める登録の完了時に、サービス利用契約が登録ユーザーと運営者の間に成立し、登録ユーザーは本サービスを本規約に従い利用することができるようになります。

    運営者は、登録申請者が以下の各号のいずれかの内容に該当する場合は、登録及び再登録を拒否することがあり、またその理由について一切開示義務を負いません。

        運営者に提供した登録事項の一部または全てに虚偽、誤記または記載漏れがあった場合

        未成年者、成年被後見人、被保佐人又は被補助人のいずれかであり、法定代理人、後見人、保佐人または補助人の同意等を得ていなかった場合

        反社会的勢力等 (暴力団、暴力団員、右翼団体、反社会的勢力、その他これに準ずる者を意味します。以下同じ。) である、または資金提供その他を通じて反社会的勢力等の維持、運営、経営への協力もしくは関与するなど、反社会的勢力等との何らかの交流もしくは関与を行っていると運営者が判断した場合

        登録希望者が過去運営者との契約に違反した者またはその関係者であると運営者が判断した場合

        第9条 (登録抹消等) に定める措置を受けたことがある場合

        その他、運営者が登録を適当でないと判断した場合

第4条 (登録事項の変更)

    登録ユーザーは、登録事項に変更があった場合、運営者の定める方法により当該変更事項を遅滞なく運営者に通知するものとします。

第5条 (パスワードの管理)

    登録ユーザーは、自己の責任において、本サービスに関するパスワードを適切に管理及び保管するものとし、これを第三者に利用させること、または貸与、譲渡、名義変更、売買等をしてはならないものとします。

    パスワードの管理不十分、使用上の過誤、第三者の使用等によって生じた損害に関する責任は登録ユーザーが負うものとし、運営者は一切の責任を負いません。

第6条 (禁止事項)

    登録ユーザーは、本サービスの利用にあたり、以下の内容のいずれかに該当する行為または該当すると運営者が判断する行為をしてはなりません。

    法令に違反する行為または犯罪行為に関連する行為

    運営者、本サービスの他の利用者またはその他の第三者に対する詐欺または脅迫行為

        公序良俗に反する行為

        運営者、本サービスの他の利用者またはその他の第三者の知的財産権、肖像権、プライバシーの権利、名誉、その他の権利または利益を侵害する行為

        本サービスを通じ、以下に該当し、または該当すると運営者が判断する情報を運営者または本サービスの他の利用者に送信すること

        過度に暴力的または残虐な表現を含む情報

        コンピュータウィルスなどの有害なコンピュータプログラムを含む情報

        運営者、本サービスの他の利用者またはその他の第三者の名誉または信用を毀損する表現を含む情報

        過度にわいせつな表現を含む情報

        差別を助長する表現を含む情報

        自殺、自傷行為を助長する表現を含む情報

        薬物の不適切な利用を助長する表現を含む情報

        反社会的な表現を含む情報

        チェーンメールなどの第三者への情報の拡散を求める情報

        他人に不快感を与える表現を含む情報

        面識のない異性との出会いを目的とした情報

    本サービスのネットワークまたはシステムなどに過度な負荷をかける行為

    本サービスの運営を妨害する恐れのある行為

    運営者が管理するネットワークまたはシステムなどへの不正なアクセス、または不正なアクセスを試みる行為

    第三者に成りすます行為

    本サービスの他の利用者のユーザー名またはパスワードを利用する行為

    運営者が事前に許諾しない本サービス上での宣伝、広告、勧誘、または営業行為

    本サービスの他の利用者の情報を収集する行為

    運営者、本サービスの他の利用者またはその他の第三者に不利益、損害、不快感を与える行為

    反社会的勢力などに利益を供与する行為

    面識のない異性との出会いを目的とした行為

    前各号の行為を直接または間接的に成す、または容易にする行為

    その他、運営者が不適切と判断する行為

第7条 (本サービスの停止等)

    運営者は、以下のいずれかに該当する場合には、登録ユーザーに事前に通知することなく、本サービスの全てまたは一部の提供を停止または中断することができるものとします。

        本サービスに使用されるコンピュータシステムの点検または保守作業を緊急に行う場合

        コンピュータ、通信回線などが事故により停止した場合

        地震、落雷、火災、風水害、停電、天災地変などの不可抗力により本サービスの運営ができなくなった場合

        その他、運営者が停止または中断を必要と判断した場合

    運営者は、本条に基づき運営者が行った措置に基づき登録ユーザーに生じた損害について一切の責任を負いません。

第8条 (権利帰属)

    本サービスに関する知的財産権は全て運営者または運営者にライセンスを許諾している者に帰属しており、本規約に基づく本サービスの利用許諾は、本サービスに関する運営者または運営者にライセンスを許諾している者の知的財産権の使用許諾を意味するものではありません。

    登録ユーザーは、投稿データについて、自らが投稿その他送信することについての適法な権利を有していること、および投稿データが第三者の権利を侵害していないことについて、運営者に対し表明し、保証するものとします。

    登録ユーザーは、投稿データについて、運営者に対し、世界的、非独占的、無償、サブライセンス可能かつ譲渡可能な使用、複製、配布、派生著作物の作成、表示及び実行に関するライセンスを付与します。また、他の登録ユーザーに対しても、本サービスを利用してユーザーが投稿その他送信した投稿データの使用、複製、配布、派生著作物を作成、表示及び実行することについての非独占的なライセンスを付与します。

    登録ユーザーは、運営者及び運営者から権利を承継しまたは許諾された者に対して著作者人格権を行使しないことに同意するものとします。

第9条 (登録抹消等)

    運営者は、登録ユーザーが、以下の各号のいずれかの事由に該当する場合は、事前に通知又は催告することなく、投稿データの削除もしくは当該登録ユーザーによる本サービスの利用の一時停止、またはユーザー登録の抹消、もしくはサービス利用契約を解除することができます。

        本規約のいずれかの条項に違反した場合

        登録事項に虚偽の事実があることが判明した場合

        運営者からの問い合わせその他の回答を求める連絡に対して30日間以上応答がない場合

        第3条 (登録) 第4項各号に該当する場合

        その他、運営者が本サービスの利用、登録ユーザーとしての登録、またはサービス利用契約の継続を適当でないと判断した場合

    運営者は、本条に基づき運営者が行った行為により登録ユーザーに生じた損害について一切の責任を負いません。

第10条 (退会)

    登録ユーザーは、運営者所定の方法で運営者に通知することにより、本サービスから退会し、自己の登録ユーザーとしての登録を抹消することができます。

    退会後の利用者情報の取扱いについては、第14条 (利用者情報の取扱い) の規定に従うものとします。

第11条 (本サービスの内容の変更、終了)

    運営者は、運営者の都合により、本サービスの内容を変更し、または提供を終了することができます。運営者が本サービスの提供を終了する場合、運営者は登録ユーザーに事前に通知するものとします。

    運営者は、本条に基づき、運営者が行った措置によって登録ユーザーに生じた損害について一切の責任を負いません。

第12条 (保証の否認及び免責)

    運営者は、本サービスが登録ユーザーの特定の目的に適合すること、期待する機能・製品的価値・正確性・有用性を有すること、登録ユーザーによる本サービスの利用が登録ユーザーに適用のある法令または業界団体の内部規則などに適合すること、および不具合が生じないことについて、何ら保証するものではありません。

    運営者は、運営者による本サービスの提供の中断、停止、終了、利用不能または変更、登録ユーザーが本サービスに送信したメッセージまたは情報の削除または消失、ユーザー登録の抹消、本サービスの利用による登録データの消失または機器の故障もしくは損傷、その他本サービスに関して登録ユーザーが被った損害 (以下「ユーザー損害」と言います。) につき、賠償する責任を一切負わないものとします。

    何らかの理由により運営者が責任を負う場合であっても、運営者は、ユーザー損害につき、付随的損害、間接損害、特別損害、将来の損害及び逸失利益にかかる損害については、賠償する責任を負わないものとします。

    本サービスに関連して登録ユーザーと他の登録ユーザーまたは第三者との間において生じた取引、連絡、紛争等については、運営者は一切責任を負いません。

第13条 (秘密保持)

    登録ユーザーは、本サービスに関連して運営者が登録ユーザーに対して秘密に取り扱うことを求めて開示した非公知の情報については、特定の事情がある場合は運営者に断りなく公開することができる。

第14条 (利用者情報の取扱い)

    運営者による登録ユーザーの利用者情報の取扱いについては、別途プライバシーポリシーの定めによるものとし、登録ユーザーはこのプライバシーポリシーに従って運営者がユーザーの利用者情報を取扱うことについて同意するものとします。

    運営者は、登録ユーザーが運営者に提供した情報、データ等を、個人を特定できない形での統計的な情報として、運営者の裁量で、利用及び公開することができるものとし、ユーザーはこれに異議を唱えないものとします。

第15条 (本規約等の変更)

    運営者は、本規約を変更できるものとします。運営者は、本規約を変更した場合には、登録ユーザーに当該変更内容を通知するものとし、当該変更内容の通知後、登録ユーザーが本サービスを利用した場合または運営者の定める期間内に登録抹消の手続きをとらなかった場合には、登録ユーザーは、本規約の変更に同意したものとみなします。

第16条 (連絡/通知)

    本サービスに関する問い合わせその他登録ユーザーから運営者に対する連絡または通知、及び本規約の変更に関する通知その他運営者から登録ユーザーに対する連絡または通知は、運営者の定める方法で行うものとします。

第17条 (利用契約上の地位の譲渡等)

    登録ユーザーは、運営者の書面による事前の承諾なく、利用契約上の地位または本規約に基づく権利もしくは義務につき、第三者に対し、譲渡、移転、担保設定、その他の処分をすることはできません。

    運営者は本サービスに関係する事業を他に譲渡した場合には、当該事業譲渡に伴い利用契約上の地位、本規約に基づく権利及び義務並びに登録ユーザーの登録事項その他の顧客情報を当該事業譲渡の譲受人に譲渡することができるものとし、登録ユーザーは、譲渡につき本項において予め同意したものとします。

第18条 (分離可能性)

    本規約のいずれかの条項またはその一部が、消費者契約法その他の法令等により無効または執行不能と判断された場合であっても、本規約の残りの規定及び一部が無効または執行不能と判断された規定の残りの部分は、継続して完全に効力を有するものとします。

第19条 (準拠法及び管轄裁判所)

    本規約及びサービス利用契約の準拠法は日本法とします。なお、本サービスにおいて物品の売買が発生する場合であっても、国際物品売買契約に関する国際連合条約の適用を排除することに合意します。

    本規約に起因、または関連する一切の紛争については、東京地方裁判所を第一審の専属的合意管轄裁判所とします。

\section{Gamma社}
第1条(適用関係)

    本規約は、本サービスの提供条件及び本サービスの利用に関する当社と本サービス利用者との間の権利義務関係を定めることを目的とし、本サービス利用者と当社との間の本サービスの利用に関わる一切の関係に適用されます。

    当社が、本サービスの一部を構成する個別のウェブサイトや個別のサービスに関連して作成、配布、公開および提示する書類上に掲載する本サービスの一部または全部の利用に関するルール規約は、本規約に準ずるものとして同様に取り扱うものとします。

    本規約の内容と、前項のルール規約その他の本規約外における本サービスの一部または全部の説明等とが異なる場合は、当該ルール規約上で本規約より優先されるとの明示の定めがある以外は、本規約が優先適用されます。

第2条(定義)

本規約において使用する以下の用語は、前文に定めているものの外、以下に定める意味を有するものとします。

    「サービス利用契約」とは、本規約及び当社と本サービス利用者の間で締結するその他の本サービスを利用する上での契約を意味します。

    「知的財産権」とは、著作権、著作者人格権、著作隣接権、パブリシティ権、特許権、実用新案権、意匠権、商標権、商号権、その他の権利(それらの権利を取得し、またはそれらの権利につき登録等を出願する権利を含みます。)を意味します。

    「利用者データ」とは、本サービス利用者が本サービスを利用して提供その他送信または伝達または送付するコンテンツ(文章、画像、映像、音声その他のデータを含みますがこれらに限りません。また電子的データの他紙面や電話での伝達記録手段によるものを含みますがこれらに限りません。)を意味します。

    「当社ウェブサイト」とは、当社が単独または第三者と共同で運営するウェブサイトをいいます。

    「本サービス」とは、インターネット上及びインターネットに関連して運営提供される当社のサービス全般を意味し、物品の販売も含みます。

    「本サービス利用者」とは、当社ウェブサイト利用者及びウェブサイト利用以外のそれに関連するサービスを利用する個人または法人を意味します。

第3条(本サービスの利用関係)

    本サービス利用者は、本規約に同意した上で、当該利用者の特定のデータを、当社の定める方法により当社に提供することができます。

    当社は、本サービス利用者が、以下の各号のいずれかに該当する場合は、本サービス利用の継続を拒否することがあります。この場合、当社は利用継続の拒否理由について、開示説明の義務を負いません。

        当社に提供された利用者データに、事実とは異なるデータがあった場合

        本サービス利用者が、未成年者、成年被後見人、被保佐人又は被補助人のいずれかに該当し、法定代理人、後見人、保佐人または補助人の承認同意等を得ずに本サイト及び本サービスを利用した場合

        反社会的勢力(「暴力団、暴力団員、右翼団体、反社会的勢力、その他これに準ずる者」を意味します。以下同じ。)に該当する、又は、資金提供その他を通じて反社会的勢力等の維持、運営、経営に協力する等反社会的勢力等の活動に関与するなど反社会的勢力等との何らかの交流関係があると当社が判断した場合

        過去当社との契約に違反した者、又は、その関係者であると当社が判断した場合

        第10条に定める措置を受けたことがある場合

        その他、当社が利用を継続するに適当でないと判断した場合

第4条(利用者データの変更)

    本サービス利用者は、当該利用者データに変更があった場合、各個別のウェブサイト上で当社が定める方法により、変更事項を遅滞なく当社に通知するものとします。

第5条(パスワード及びユーザーIDの管理)

    本サービスを構成する個別のサービスによっては、当該サービスを利用する上で使用するパスワード及びユーザーIDを、本サービス利用者自身で設定する必要があります。その場合、本サービス利用者はそのパスワード及びユーザーIDを自身で適切に管理及び保管するものとし、これを第三者に利用させ、又は貸与、譲渡、名義変更、売買等をしてはなりません。

    本サービス利用者によるパスワードまたはユーザーIDの管理不十分、使用上の過誤、第三者使用の許諾等によって生じた損害に関する責任は、本サービス利用者が負うものとし、当社は一切の責任を負いません。

第6条(対価の定めがある場合)

    本サービスを構成する個別のサービスにサービス利用の対価の定めがある場合には、本サービス利用者は、その個別のサービスに関するウェブサイトその他媒体で当社が定め表示する利用料金その他の金員を、当社が指定する支払い方法その他の取引条件に従って当社に支払うものとします。

    本サービス利用者が利用料金その他の金員の支払いを遅滞した場合、本サービス利用者は年14.6%の割合による遅延損害金を当社に支払うものとします。

第7条(禁止事項)

本サービス利用者は、以下の各号のいずれかに該当する行為又は該当すると当社が判断する行為をしてはなりません。

    犯罪行為、法令に違反する行為及びこれらに関係する行為

    当社、本サービスの他の利用者又はその他の第三者に対する詐欺又は脅迫及びそれらの行為に準ずる行為

    当社、本サービスの他の利用者又はその他の第三者の知的財産権、肖像権、プライバシー権、名誉権、その他の権利又は利益を侵害する行為

        本サービスを通じ、以下に該当すると当社が判断する情報を本サービスの他の利用者に送信する行為

        過度に暴力的または残虐な表現を含む情報

        コンピュータウイルスその他の有害なコンピュータプログラムを含む情報

        当社、本サービスの他の利用者又はその他の第三者の名誉又は信用を棄損する表現を含む情報

        薬物の不適切な利用を助長する表現を含む情報

        反社会的な表現を含む情報

        チェーンメール等の第三者への情報の拡散を求める情報

        他人に過度な不快感を与える表現を含む情報

        面識のない異性との出会いを目的とした情報

    本サービスのネットワークまたはシステム等に過度の負荷をかける等本サービスの運営を妨害する恐れのある行為

    当社のネットワークまたはシステム等に不正にアクセスし、又は不正なアクセスを試みる行為

    本サービスの他の利用者のIDやパスワードを利用し、その者であるかのごとく騙る行為

    当社が事前に許諾しない本サービス上での宣伝、広告、勧誘、又は営業行為

    他の本サービス利用者の情報の収集をする行為

    当社、他の本サービス利用者又はその他の第三者に不利益、損害、不快感を与える行為

    当社が本サービスの一部を構成する個別のウェブサイトや個別のサービスに関連して作成配布公開提示する書類上で掲載する本サービスの一部または全部の利用に関するルールに抵触する行為

    反社会的勢力等への利益供与およびこれに準じる行為

    面識のない異性との出会いを目的とした行為

    前各号の行為を直接又は間接に惹起し、又は容易にする行為

    その他、公序良俗等に反するものとして当社が不適切と判断する行為

第8条(本サービスの停止等)

当社は、以下のいずれかに該当する場合には、本サービス利用者に事前に通知することなく、本サービスの全部または一部の提供を停止又は中断することができるものとします。

    本サービスに係るコンピュータ及びネットワークシステムの点検又は保守作業を行う場合

    コンピュータ、通信回線等が不可抗力や事故により利用不能となった場合

    地震、落雷、火災、風水害、停電、天災事変等の不可抗力により本サービスの運営ができなくなった場合

    その他、当社が停止又は中断が必要と判断した場合

第9条(権利帰属)

    本サイト及び本サービスに関する知的財産権は、すべて当社又は当社にライセンスを許諾している者に帰属しています。本規約に基づく本サービスの利用許諾は、本サイト又は本サービスに関する当社又は当社にライセンスを許諾している者に帰属している知的財産権の使用許諾を意味しません。

    本サービス利用者が当社に利用者データを提供した場合、当該利用者は当該データの提供に関して、適法に権限権利が帰属していること、及び、当該利用者データが第三者の権利を侵害していないこと、を当社に対し表明し、保証するものとします。

    本サービス利用者が当社に利用者データを提供した場合、当該利用者データについては、当社に対し、世界的、非独占的、無償、サブライセンス可能かつ譲渡可能な使用、複製、配布、派生著作物の作成、表示および実行に関するライセンスが付与されます。また、他の本サービス利用者に対しても、当該利用者データの使用、複製、配布、派生著作物を作成、表示及び実行することについての非独占的なライセンスが付与されます。

    本サービス利用者は、当社及び当社から権利を承継しまたは許諾された者に対して、著作者人格権を行使しないことに同意するものとします。

第10条(登録抹消及び精算義務)

    当社は、本サービス利用者が、以下の各号のいずれかの事由に該当する場合は、事前に通知または催告することなく、利用者データを削除しもしくは当該本サービス利用者について本サービスの利用を一時的に停止し、または本サービス利用者としての登録を抹消、もしくはサービス利用契約を解除することができます。

        本規約のいずれかの条項に違反した場合

        登録等当社に提出された事項に虚偽の事実があることが判明した場合

        24ヶ月以上本サービスの利用がない場合

        当社からの問いあわせその他の回答を求める連絡に対して10日間以上応答がない場合

        その他、当社が本サービスの利用、本サービス利用者としての登録、またはサービス利用契約の継続を適当でないと判断した場合

    本サービス利用者が、前項により登録を抹消又はサービス利用契約を解除された場合、当該利用者が当社に対して負う一切の債務は当然に期限の利益を失い、直ちに当社に全ての債務の支払を行わなければなりません。

    本サービス利用者は、当社が本サービスの一部として提供する個別のサービス上で定められた当社所定の方法により、当社に通知することで、本サービス利用者としての登録を抹消、もしくはサービス利用契約を解除することができます。

    前項の場合、本条第2項が準用されます。

    本サービス利用者としての登録を抹消又はサービス利用契約を解除された場合の利用者データの取扱いについては、第14条の規定に従うものとします。

第11条(本サービスの内容の変更または終了)

    当社は、本サービスの内容を変更し、または、提供を終了することができます。

    当社が本サービスの提供を終了する場合、当社は本サービス利用者に事前にその旨を公開しなくてよいものとします。

第12条(保証の否認及び免責)

    当社は、本サイト及び本サービス(本サービスに情報が掲載されている事業者(以下「掲載事業者」といいます。)によるサービス提供を含む。)が、本サービス利用者の特定の目的に適合すること、本サービス利用者の期待する機能商品的価値正確性有用性を有すること、本サービス利用者に適用のある法令や業界団体の内部規則等に適合すること、及び、本サービス利用者に何らかの不都合が生じないこと、について何ら保証いたしません。

    本サービスの利用に関して、当社の責に帰すべき事由により損害賠償責任を当社が負う場合の範囲は、本サービス利用者に生じた通常かつ直接の損害の範囲に限り、当社の予見の有無にかかわらず特別の事情から生じた特別損害については責任を負いません。また、損害賠償額は、本サービス利用者が当社にお支払いいただいた過去3ヶ月間の対価合計の相当額を上限とします。ただし、当社に重過失がある場合はこの限りではありません。

    本サービス利用者は、掲載事業者に関するものとして本サイトに掲載されている情報が当該掲載事業者の責任において掲載されているものであることを理解したうえで、掲載事業者と交渉、契約等を行うに際しては、自己の責任において直接それを行うものとし、本サイトまたは本サービスに関連して、本サービス利用者と他の本サービス利用者または提携事業者その他の第三者との間において生じた取引、連絡、紛争等については、当社は関与せず、一切の責任も負いません。

第13条(秘密保持)

    当社は、本サービス利用者に、本サービスに関連した情報を開示することがあります。

    この場合、秘密(非開示)情報として開示した情報については、法令の手続きによる場合、又は、当社による書面の事前承認がある場合を除き、当該情報は第三者に開示できません。

第14条(利用者データの取扱い)

    本サービスを利用して提供された利用者データの取り扱い、当社のプライバシーポリシーに従い、本サービス利用者はこの取扱いについて同意するものとします。

    当社は、本サービス利用者が当社に提供した利用者データを、個人を特定できない形での統計的な情報として、当社の裁量で利用及び公開することができるものとし、本サービス利用者はこの取り扱いについて同意するものとします。

    当社は、提供された利用者データが当社又は他の本サービス利用者又は第三者の権利侵害に当たるなど本規約に違反すると判断した場合には、利用者データの提供者や本サービス利用者への通知や合意を得ることなく、その削除ができるものとします。この場合、利用者データの提供者や当該本サービス利用者に対しては、第10条2項が準用されます。

第15条(補充規定等)

    当社は、本規約を必要に応じて変更し、また、本規約を補充する規定(以下「補充規定」といいます)を定立することができます。

    本規約の変更または補充規定は、変更後の本規約または補充規定を当社の運営するウェブサイトに掲示公開した際に明示された期日にその効力を生じるものとします。特に期日の定めがなされていなかった場合は、当社の運営するウェブサイトに本規約の変更または補充規定が掲示公開された日を効力発生日とします。

    別段の定めがない限り、規約の変更補充後、本サービスの利用を継続することにより、当該変更補充について、本サービス利用者は同意承認したものとします。

    本サービス利用者が、本規約の変更または補充規定に同意承認しない場合、ただちに本サービスの利用を停止してください。

第16条(連絡/通知)

    本サービスに関する問い合わせその他当社に対する連絡、当社への通知等に対する当社からの連絡、および、本規約の変更や補充に関する掲示公開その他当社から本サービス利用者に対する通知は、本サイトおよび本サービスの一部を構成する個別のウェブサイトや個別のサービスに関連して作成配布公開提示する書類上で掲載する本サービスの一部または全部の利用に関するポリシー規約等上で当社が定める方法によって行います。

第17条(サービス利用契約上の地位の譲渡等)

    本サービス利用者は、当社の書面による事前の承諾なく、サービス利用契約上の地位または本規約に基づく権利もしくは義務につき、第三者に対し、譲渡、移転、その他の処分をすることはできません。

    当社が本サービスにかかる事業を他社に譲渡した場合、当該事業譲渡に伴いサービス利用契約上の地位、本規約に基づく権利及び義務並びに本サービス利用者の利用者データその他の本サービス利用者の情報を、当該事業譲渡の譲受人に譲渡することができるものとし、本サービス利用者は、かかる譲渡につき予め同意するものとします。

    なお、本項に定める事業譲渡には、通常の営業譲渡や事業譲渡のみならず、会社分割その他により本サービスを運営する事業が移転するあらゆる場合を含みます。

第18条(分離可能性)

    本規約のいずれかの条項またはその一部が、消費者契約法、電子消費者契約法及びその他の関係法令等により、無効と判断された場合でも、それ以外の部分は、継続して効力を有するものとします。

第19条(準拠法及び管轄裁判所)

    本規約及び本サービス関係の準拠法は日本法とします。 また、本サービスにおいて物品の売買が発生する場合であっても、国際物品売買契約に関する国際連合条約の適用が排除されることに合意します。

    本規約または本サービスに起因し、または関連する一切の紛争については、東京地方裁判所または東京簡易裁判所を第一審の専属的合意管轄裁判所とします。

\section{Delta社}
第1条 本利用規約について

    「Delta利用規約」(以下「本利用規約」といいます。)は、Delta株式会社(以下「当社」といいます。)が当社のウェブサイト及びアプリケーション(以下「本サイト」といいます。)において提供するすべてのサービス(以下「本サービス」といいます。)の利用条件を定めるものです。本サービスをご利用される場合には、利用者は本利用規約に同意したものとみなされ、本利用規約を内容とする本サービスの利用契約が成立したものとみなします。

    当社は、個別のサービスにおいて個別の規約やガイドライン(以下「個別規約」といいます。)を定めています。個別規約も名称の如何にかかわらず本利用規約の一部を構成するものとします(以下、本利用規約と個別規約をあわせて「本利用規約」といい、本利用規約のうち個別規約を含まない部分に言及する場合は、「Delta利用規約」といいます。)。なお、個別サービスにおいては、Delta利用規約のほか当該個別のサービスにおいて定められる個別規約も適用されますので、当該個別規約の内容もご確認ください。

    Delta利用規約の定めと個別規約の定めが異なる場合には、当該個別規約の定めが優先して適用されるものとします。

    当社は、民法その他の法令により認められる場合、利用者の事前の承諾を得ることなく、本利用規約を変更することができるものとします。

    本利用規約の一部の規定の全部または一部が法令に基づいて無効と判断された場合であっても、当該規定の無効部分以外の部分および本利用規約のその他の規定は有効とします。本利用規約の一部が特定の利用者との間で無効とされ、または取り消された場合でも、本利用規約はその他の利用者との関係では有効とします。

第2条 定義

本利用規約において使用する用語の意義は、次の各号に定めるとおりとします。

    「利用者」とは、ユーザー登録の有無にかかわらず、本サービスを利用する全ての方をいいます。

    「登録者」とは、ユーザー登録手続きを行い本サービス(無料機能のみならず有料機能によって提供されるサービスを含みます。)を利用する方をいいます。

    「ユーザー登録」とは、第3条第3項に基づいて本サービスのユーザーとして当社に登録されることをいいます。

    「企業・団体」とは、法人、組合、サークル等の自然人が所属する組織をいいます。

    「企業・団体登録」とは、企業・団体の情報が当社のサーバーに登録されることをいいます。

    「パスワード」とは、登録者が登録手続き時に登録し、または登録後に変更手続きを行った、登録者本人を識別するための文字および数字の列をいいます。

    「登録情報」とは、ニックネーム、メールアドレス、パスワード、郵便番号、生年月日、性別、企業・団体に関する情報等、登録者がユーザー登録または企業・団体登録の手続き時に登録した情報をいいます。

    「届出情報」とは、住所、氏名、電話番号、職業、銀行口座、クレジットカード番号等、登録者が当社に対して届け出た登録情報以外の情報をいいます。

    「取得情報」とは、利用者のIPアドレス、利用状況、履歴、位置情報、利用端末、クレジット決済ができなかった場合はその事実等、当社が取得する利用者に関するすべての情報であって、登録情報及び届出情報以外の情報をいいます。

    「個人情報」とは、個人情報の保護に関する法律に定める「個人情報」を指すものとします。

第3条 ユーザー登録

    本サービスはユーザー登録をせずにご利用いただくことも可能ですが、本サービスのうち特定のサービスまたは機能をご利用いただく場合にはユーザー登録が必要となります。

    ユーザー登録を希望する利用者(以下「申込者」といいます。)は、本利用規約に同意の上、当社所定の登録手続きを行っていただきます。

    申込者が前項の登録手続きを行った場合、当社は、前項の登録手続きを通じて入力されたお申込み内容につき必要な確認を行います。ユーザー登録は、当社が申込者の申し込みを承諾することにより完了します。なお、お申込み内容の審査を行った結果、次のいずれかに該当する場合には、当社の判断によってユーザー登録を承諾しないことがあります。

        申込者が、過去に本利用規約に違反したことを理由に当社から処分を受けた者である場合

        その他当社が不適切と判断した場合

    当社は、前項とは別に、ユーザー登録の事前事後を問わず、申込者または登録者のお申込み内容または登録情報を審査し、以下の項目に該当すると当社が判断した場合には、当社の判断により、ユーザー登録を承諾せず、または解除することができるものとします。

        申込をした時点で第17条第1項に定める処分を受けている、または過去に受けたことがある場合または第16条各項各号のいずれかに該当しもしくは該当するおそれがある場合

        お申込み内容または登録情報に故意による虚偽の記載があった場合

        申込者または登録者が未成年者、成年被後見人、被保佐人または被補助人のいずれかであり、登録手続きが成年後見人によって行われておらず、または登録手続きの際に、法定代理人、保佐人もしくは補助人の同意を得ていなかった場合

        その他当社が申込者または登録者のユーザー登録を不適当と判断した場合

第4条 登録情報等の管理

    当社は、登録情報が不正確または虚偽であったために登録者が被った一切の不利益および損害に関し、一切の責任を負わないものとします。

    登録者は、登録情報のうち本サービスへのログインに用いる情報を、自己の責任の下で管理するものとします。

    当社は、本サービスへのログイン時に送信された情報が、登録情報と一致することを所定の方法により確認した場合、当該ログインを真正な登録者のログインとみなし、登録者による利用とみなします。

第5条 個人情報の取扱い等

    当社は、当社が取得した個人情報を、別途定める「サービスプライバシーポリシー」及び個別のサービス専用のプライバシーポリシーに基づき、適切に取り扱うものとします。

    利用者は、本サービスを利用するに際し、利用者から取得する個人情報を、「サービスプライバシーポリシー」における当社が取得する個人情報の第三者に提供に関する定めに従い、第三者に提供することに同意します。

    利用者は、利用者間の識別のため、ニックネーム及びアイコン写真等が本サイト上に表示されることに同意するものとします。

第6条 届出情報の変更、ユーザー登録の解除等

    登録者は、届出情報に変更があった場合、すみやかに当社の定める手続きにより当社に届け出るものとします。この届出がない場合、当社は届出情報の変更がないものとして取り扱います。

    登録者からの届出情報の変更の届出がないために、当社からの通知、その他が遅延し、または不着、不履行であった場合、当社はその責任を負わないものとします。

    登録者は、当社所定のフォームを利用した登録解除の手続きを行うことによって、ユーザー登録を解除することができます。

第8条 連絡または通知

    当社は、登録者への連絡または通知を、当社が適切と判断した方法により行います。なお、届出情報に含まれる住所、電話番号もしくはメールアドレスその他の連絡先または登録者が利用している当社が提供するアプリケーションに当社の連絡または通知が到達した場合、登録者は当社の連絡または通知を受領したものとみなします。

    利用者は、本利用規約に別段の定めがある場合を除き、当社への連絡はお問い合わせフォームから行うものとします。当社は電話による連絡および来訪は受け付けておりません。

第9条 本サービスの提供の中断等

    当社は、以下のいずれかの事由が生じた場合には、利用者に事前に通知することなく、一時的に本サービスの全部または一部の提供を中断することがあります。

        本サービスを提供するための通信設備等の定期的な保守点検を行う場合または点検を緊急に行う場合

        地震、噴火、洪水、津波等の天災により本サービスの提供ができなくなった場合

        その他、運用上または技術上、当社が本サービスの提供の一時的な中断を必要と判断した場合

    当社が必要と判断した場合には、事前に通知することなくいつでも本サービスの内容を変更し、または本サービスの提供を停止もしくは中止することができるものとします。

    当社は、第1項各号のいずれかまたはその他の事由により本サービスの全部または一部の提供に遅延もしくは中断が発生しても、これに起因する利用者または第三者が被った損害に関し、本利用規約で特に定める場合を除き、一切の責任を負いません。

    当社が本サービスの内容を変更し、または本サービスの提供を停止もしくは中止した場合であっても、本利用規約で特に定める場合を除き、利用者に対して一切責任を負わないものとします。

第10条 利用環境の整備

    利用者は、本サービスを利用するために必要な通信機器、ソフトウェアその他これらに付随して必要となる全ての機器を、自己の費用と責任において準備し、利用可能な状態に置くものとします。また、本サービスのご利用にあたっては、自己の費用と責任において、利用者が任意に選択し、電気通信サービスまたは電気通信回線を経由してインターネットに接続するものとします。

    利用者は、関係官庁等が提供する情報を参考にして、自己の利用環境に応じ、コンピュータ・ウィルスの感染、不正アクセスおよび情報漏洩の防止等セキュリティを保持するものとします。

第11条 自己責任の原則

    利用者は、利用者自身の自己責任において本サービスを利用するものとし、本サービスを利用してなされた一切の行為およびその結果についてその責任を負うものとします。

    利用者は、本サービスのご利用に際し、他の利用者その他の第三者および当社に損害または不利益を与えた場合、自己の責任と費用においてこれを解決するものとします。

第12条 知的財産権等

    利用者は、利用者が送信(発信)したコンテンツにつき、当社に対して、当社または当社の指定する者が当該コンテンツを日本国内外問わず対価の支払いなく非独占的にいかなる制約も受けずに自由に使用する(複製、公開、送信、頒布、譲渡、貸与、翻訳、翻案を含みます。)権利(サブライセンス権も含みます。)を、当該コンテンツに係る著作権その他一切の権利の存続期間が満了するまでの間、許諾したものとみなされるものとし、これをあらかじめ承諾します。利用者は当社および当社の指定する者に対して、当該コンテンツに係る著作者人格権を保有していたとしても、当該権利を行使しないものとします。

    利用者が送信(発信)したコンテンツ、利用者によるサービスの利用・接続・規約違反、利用者による第三者への権利侵害に起因または関連して生じたすべてのクレームや請求について、利用者の責任と費用においてこれを解決するものとします。

    前項のクレームや請求への対応に関連して当社に費用が発生した場合または賠償金等の支払いを行った場合は、当該費用および賠償金、当社が支払った弁護士費用等を当該利用者の負担とし、当社は、当該利用者にこれらの合計額の支払いを請求できるものとします。

    利用者は、利用者が送信(発信)したコンテンツについて、当社に保存義務がないことを認識し、必要なコンテンツは適宜バックアップをとるものとします。

    当社は、利用者が送信(発信)したコンテンツを、運営上必要に応じて閲覧することができ、規約に抵触すると判断した場合には、利用者への事前の通知なしに、当該コンテンツの全部または一部を非公開すること、または削除することができるものとします。

第13条 当社の財産権

    利用者が送信(発信)したコンテンツおよび情報を除き、本サービスに含まれる一切のコンテンツおよび情報に関する財産権は当社に帰属します。

    本サービスまたは広告中に掲載・提供されているコンテンツは、著作権法、商標法、意匠法等により保護されております。

    本サービスおよび本サービスに関連して使用されているすべてのソフトウェアは、知的財産権に関する法令等により保護されている財産権および営業秘密を含んでおります。

第14条 禁止事項

    利用者は、本サービスの利用に際して、以下の行為を行ってはならないものとします。

        当社、他の利用者もしくはその他の第三者(以下「他者」といいます。)の著作権、商標権等の知的財産権を侵害する行為、または侵害するおそれのある行為

        他者の財産、プライバシーもしくは肖像権を侵害する行為、または侵害するおそれのある行為

        一人の利用者が複数の登録者の地位を保有する行為(但し、当社が別途指定する場合を除きます。)または一つの登録者の地位を複数人で共同して保有する行為

        アクセス可能な本サービスのコンテンツもしくは情報または他者のコンテンツもしくは情報を改ざん、消去する行為

        他者に対し、無断で、広告・宣伝・勧誘等の電子メールもしくは嫌悪感を抱く電子メール(そのおそれのある電子メールを含みます。)を送信する行為、他者のメール受信を妨害する行為、連鎖的なメール転送を依頼する行為または当該依頼に応じて転送する行為

        通常に本サービスを利用する行為を超えてサーバーに負荷をかける行為もしくはそれを助長するような行為、その他本サービスの運営・提供もしくは他の利用者による本サービスの利用を妨害し、またはそれらに支障をきたす行為

        本サービスによって提供される機能を複製、修正、転載、改変、変更、リバースエンジニアリング、逆アセンブル、逆コンパイル、翻訳あるいは解析する行為

        本人の同意を得ることなく、または詐欺的な手段(いわゆるフィッシングおよびこれに類する手段を含みます。)により他者の登録情報を取得する行為

        本サービスの全部または一部を営利目的で、使用方法を問わず利用する行為(それらの準備を目的とした行為も含みます。但し、当社が認めた場合は除きます。)

        法令に基づき監督官庁等への届出、許認可の取得等の手続きが義務づけられている場合に、当該手続きを履行せずに本サービスを利用する行為、その他当該法令に違反し、または違反するおそれのある行為

        上記各号の他、法令もしくは本利用規約(当社が別途定めるガイドラインを含みます。)に違反する行為、または公序良俗に違反する行為(暴力を助長し、誘発するおそれのある情報もしくは残虐な映像を送信もしくは表示する行為や心中の仲間を募る行為等を含みます。)

        上記各号のいずれかに該当する行為(当該行為を他者が行っている場合を含みます。)が見られるデータ等へ当該行為を助長する目的でリンクを張る行為

        その他当社が利用者として不適当と判断した行為

第15条 ユーザー登録解除等

    当社は、利用者の行為が本利用規約に反すると判断した場合または利用者が前条第2項に定める者に該当すると判断した場合には、当社の判断により、当該利用者に何ら通知することなくして、当社が本サービスを通じて送信(発信)されたコンテンツの削除および変更ならびにサービスの一時停止、ユーザー登録の解除(登録者の地位の停止を含みます。)、本サイトへのアクセス拒否をすることができるものとします。

    当社が前項の処分をしたときは、当社所定の方法でその旨を利用者に通知することとします。

    前項に定める当社からの通知が、利用者の事情によって当該利用者に到達しなかった場合、当社からの通知はその発信時に当該利用者に到達したものとみなします。

    本条の定めに従ってなされた当社の処分に関する質問、苦情は一切受け付けておりません。

第16条 利用制限

    当社は、利用者が以下のいずれかに該当する場合には、当該利用者の承諾を得ることなく、当該利用者の本サービスの利用を制限することがあります。

        ワーム型ウィルスの感染、大量送信メールの経路等により、当該登録者が関与することにより第三者に被害が及ぶおそれがあると判断した場合

        届出情報に含まれる電話番号宛の電話もしくはメールアドレス宛の電子メールまたは本サービス上の通知等による連絡がとれない場合

        上記各号の他、当社が緊急性が高いと認めた場合

    当社が前項に基づき利用者の本サービスの利用を制限したことにより、当該利用者が本サービスを利用できず、これにより損害が発生したとしても、当社は一切責任を負いません。

第17条 免責

    当社は、本サービスの利用により発生した利用者の損害については、一切の賠償責任を負いません。

    利用者が、本サービスを利用することにより、第三者に対し損害を与えた場合、利用者は自己の費用と責任においてこれを賠償するものとします。

    当社は本サービスに発生した不具合、エラー、障害等により本サービスが利用できないことによって引き起こされた損害について一切の賠償責任を負いません 。

    本サービスならびに本サイト上のコンテンツおよび情報は、当社がその時点で提供可能なものとします。当社は提供する情報、利用者が登録・送信(発信)する文章その他のコンテンツおよびソフトウェア等の情報について、その完全性、正確性、適用性、有用性、利用可能性、安全性、確実性等につきいかなる保証も一切しません。

    当社は、利用者に対して、適宜情報提供やアドバイスを行うことがありますが、その結果について責任を負わないものとします。

    本サービスが何らかの外的要因により、データ破損等をした場合、当社はその責任を負いません。

    利用者との間の本利用規約に基づく契約が消費者契約法第2条第3項の消費者契約に該当する場合には、本利用規約のうち、当社の責任を完全に免責する規定は適用されないものとします。本利用規約に基づく契約が消費者契約に該当し、かつ、当社が債務不履行または不法行為に基づき損害賠償責任を負う場合については、当社に故意または重過失がある場合を除いて、当社は、当該利用者が直接かつ現実に被った損害を上限として損害賠償責任を負うものとし、特別な事情から生じた損害等(損害発生につき予見し、または予見し得た場合を含みます。)については責任を負わないものとします。

第18条 準拠法

    本規約は、日本法に準拠し、解釈されるものとします。

第19条 裁判管轄

    利用者と当社との間で訴訟の必要が生じた場合、東京地方裁判所を第一審の専属的合意管轄裁判所とします。

\section{Epsilon社}
第1条(本規約の適用)

    本規約は、Epsilon株式会社(以下、「当社」という。)が提供する各種サービス(以下、「本サービス」という。)を、第4条に規定する会員(以下、「会員」という。)及びユーザー(以下、「ユーザー」という。)が利用する場合に、共通に適用されます。

    本サービスの会員及びユーザーは、本規約を遵守するものとします。

第2条(本規約の範囲)

    当社が本サービス上に表示する本サービスの利用方法、利用条件、利用環境等 に関する諸規定は、名称の如何にかかわらず本規約の一部を構成するものとします。

第3条(本規約の変更)

    当社は、次の各号のいずれかに該当する場合、会員及びユーザーの承諾を得ることなく、本規約の内容を変更すること(本規約に新たな内容を追加することを含む。)ができるものとします。

        利用規約の変更が、会員及びユーザーの一般の利益に適合するとき。

        利用規約の変更が、契約をした目的に反せず、かつ、変更の必要性、変更後の内容の相当性、変更の内容その他の変更に係る事情に照らして合理的なものであるとき。

    当社は前項による利用規約の変更にあたり、変更後の利用規約の効力発生日の1か月前までに、当社ホームページ上に利用規約を変更する旨及び変更後の本規約とその効力発生日を表示します。

第4条(会員及びユーザー)

    会員とは、当社に本サービスの利用を申し込み、当社がその申し込みを審査、承認して会員資格を付与した者(個人又は法人その他の団体)、又は当社が別途定める方法により会員資格を付与した者(個人又は法人その他の団体)をいいます。

    ユーザーとは、次条第1項第1号のユーザー認証方式が選択された場合は、会員が指定する本サービスの利用者であって、当社がその指定を審査、承認してユーザー資格を付与した者をいい、同項第2号のユーザー認証方式が選択された場合は、当該固定グローバルIPアドレスから本サービスを利用する利用者をいいます。

    会員及びユーザーは、本サービスの利用開始の時点で本規約の内容を承諾したものとみなします。

第5条(同時利用制限)

    前条第1項各号のいずれかのユーザー認証方式を選択する場合においても、本サービスに同時ログインすることができる件数は、申込時の契約数と同数とします。

第6条(自己責任の原則)

    当社は、本サービスの利用により発生した会員及びユーザーの損害については、損害賠償義務その他いかなる責任も負わないものとします。

    会員及びユーザーが本サービスの利用の際、第三者に対して損害を与えた場合、会員及びユーザーは、自己の責任と費用をもって解決し、当社に何ら損害を与えないものとします。

    会員又はユーザーが本規約に違反して当社に損害を与えた場合、会員及び当該ユーザーは、連帯して、当社に対して、その損害を賠償するものとします。

第7条(本サービスの利用環境)

    会員及びユーザーは、本サービスを利用するために必要な通信機器、ソフトウェア、その他これらに付随して必要となるすべての機器(以下「設備等」という。)を、自己の費用と責任において調達するものとします。また、自己の費用と責任で、任意の電気通信サービスを経由して本サービスに設備等を接続するものとします。

    会員及びユーザーは、本サービスを当社が定めた動作環境下で利用するものとします。当社の定めた動作環境以外の環境では、本サービスの全部又は一部が利用できない場合があります。

    プラグインソフトなどのダウンロードについては、会員及びユーザーの責任と費用負担で実施するものとし、当社は一切責任を負わないものとします。

    当社が定めた動作環境下の利用であっても、会員又はユーザーが保有するソフトウェア及びアプリケーション等の影響により、本サービスが誤作動・作動不良が発生した場合については、当社は一切責任を負わないものとします。

第8条(資格の付与の拒絶及び取消し)

    当社は、第4条第1項の審査の結果、会員になろうとする者が次の各号のいずれかの事由に該当することが判明した場合、会員資格の付与を拒絶することができるものとします。

        会員になろうとする者が実在しないとき。

        利用申し込みの際の申告事項に、虚偽の記載、誤記、又は記入漏れがあったとき。

        会員になろうとする者が利用料等の支払いを怠るおそれがあると当社が判断したとき。

        会員になろうとする者が、現に本サービスの利用料等の支払いを怠っているとき、又は過去にその支払いを怠ったことがあるとき。

        会員になろうとする者が未成年者、成年被後見人、被保佐人又は被補助人であり、利用申し込みの際に法定代理人、後見人、保佐人又は補助人の同意等を得ていなかったとき。

        その他当社が会員とすることを不適当と判断したとき。

    当社は、会員資格の付与後であっても、会員が前項各号のいずれかの事由に該当することが判明した場合、会員資格の付与を取り消すことができるものとします。

    前各項の規定は、ユーザー資格の付与の拒絶及び取消しについて準用するものとします。

    当社が、前各項により、会員資格又はユーザー資格の付与を取り消した場合であっても、会員は、その取消しまでの期間分の利用料等を支払うものとします。

第9条(登録内容の変更)

    会員は、当社への登録事項に変更が生じた場合は、速やかに変更内容を届け出るものとします。

    前項の届出を行わなかったことにより、会員又はユーザーが不利益を被ったとしても、当社は、一切その責任を負わないものとします。

第10条(会員資格等の譲渡等の禁止)

    会員及びユーザーは、会員資格又はユーザー資格を第三者に譲渡又は貸与することはできないものとします。

第11条(会員資格の承継等)

    相続又は法人の合併等により会員の資格の承継があるときは、会員は、承継について速やかに通知し、当社は、当該通知に従って登録内容を変更するものとします。

    当社は、会員について次の各号のいずれかの変更があったときは、その会員の構成員、従業員、業務等の同一性及び継続性が認められる場合に限り、前項の会員資格の承継があったものとみなし、前項の規定を適用します。

第12条(契約期間)

    本サービスの契約期間は、当社が会員に対し会員資格の付与を通知した日から1年間(ただし、当社と会員が契約期間について別段の合意をした場合は、その期間)とします。

    契約期間の満了の1ヶ月前までに、当社又は会員から相手方に対して、書面による別段の意思表示がなされない限り、契約期間は自動的にその契約期間が1年未満の場合は1年、1年以上の場合は、契約年数と同年数継続するものとし、途中解約はできないものとします。また、その後も同様とします。

第13条(契約終了後の処置)

    当社は、会員との契約が事由の如何を問わず終了した場合は、当社のシステム上に登録されたID等のデータ、ファイル等を削除するものとします。

第14条(本サービスの利用料)

    本サービスの料金体系、算出方法、支払い方法等は、当社が別途定めるとおりとし、随時改定することができるものとします。

    消費税等の算定の際の税率は、当該算定時に法律上有効な税率とします。

第15条(利用料等の支払い)

    会員は、利用料等の支払いその他の債務を、当社の指定する方法により、履行するものとします。

    会員が利用料等の支払いその他の債務の履行を遅滞した場合、会員は、当社に対し、年14.6%の割合の遅延損害金を支払うものとします。

第16条(禁止事項)

    会員及びユーザーが、本サービスについて、次の各号のいずれかの行為を自ら行い、又は第三者に行わせることは、固く禁止します。

        本サービスにアクセス、会員登録やサービスの利用などの行為

        スマートフォンよって本サービスにアクセスし、本サービスに関する情報を取得する行為

    当社は、前項の行為が行われたと判断した場合は、会員及びユーザーに通知し、対応を求めることがあります。

第17条(著作権、商標等の私的利用限定)

    本サービスに含まれるすべてのデータ、情報、文章、画像、ソフトウェア等一切の著作物に関する著作権は当社及び当社への情報提供者に帰属します。会員及びユーザーは、著作権法で認められた私的利用若しくは内部利用目的でのみ、本サービスを通じて入手した資料を利用することができるものとし、当社の許可なく、資料を複製し、公衆送信し、出版し、頒布する等、私的利用若しくは内部利用目的の範囲を超えて利用することはできないものとします。

    本サービスに含まれる一切の商標、サービスマーク、ロゴ等は当社の登録商標又は商標です。会員及びユーザーは私的利用若しくは内部利用目的以外で無断に利用することはできないものとします。

    会員及びユーザーは、前各項に反する行為を第三者に行わせることはできないものとします。

    ダウンロードされたデータは、会員及びユーザーが、保有する端末又は記憶媒体に保管し所持することができますが、その権利は会員及びユーザーに譲渡するものではなく、当社及び当社への情報提供者に帰属します。

第18条(本サービスの内容の変更)

    当社は、会員及びユーザーへの事前の通知なくして、本サービスのサービス内容を変更することがあります。

第19条(資料の不備についての責任)

    当社は、本サービスの資料に誤り、脱漏その他の不備のあることが発見された場合には、速やかに修正するよう努力するものとします。なお、当該不備についての当社の責任は、当該不備の修正のための合理的努力のみに限られるものとし、それ以外の責任は一切負わないものとします。

第20条(サービスの一時的な中断)

    当社は、次号のいずれかの場合には、会員及びユーザーに事前に通知すること なく、一時的に本サービスを中断することがあります。

        本サービス提供のためのシステム又は関連設備の保守を定期的又は緊急に行うとき。

        当社が利用する通信回線、電力等の提供が中断されたとき。

        火災、停電等により本サービスの提供ができなくなったとき。

        地震、噴火、洪水、津波等の天災その他の非常事態が発生し、また、そのおそれが生じたために、法令・指導により通信の制限等の要請、指示があったとき、又は当社がそれを必要と判断したとき。

        その他、運用上又は技術上当社が本サービスの一時的な中断が必要と判断したとき。

    当社は、前項各号以外の事由により、本サービスの提供の遅滞又は中断等が発生したとしても、これに起因して会員若しくはユーザー又は第三者が被った損害については、一切責任を負わないものとします。

第21条(サービスの中止)

    当社は、3ヶ月間の予告期間をもって会員及びユーザーに本サービス上にて通知の上、本サービスの提供を中止することができます。

    当社は、本サービスの提供の中止の際、前項の手続きを経ることで、中止に伴う会員若しくはユーザー又は第三者に対する損害賠償その他の責任を一切負わないものとします。

第22条(秘密の保持)

    当社は、本サービスの提供に際して知り得た会員及びユーザーの個人情報を第三者に開示又は漏洩しないものとします。ただし、次の各号のいずれかの場合においては、当社の関係会社、代理店、業務委託先その他の第三者へ会員及びユーザーの個人情報を提供又は預託する場合があることを会員及びユーザーは予め承認するものとします。

        会員及びユーザーに商品や本サービス又はそれらに関する各種情報を提供する場合

        個人を識別できない範囲内又は状態で開示する場合

        公的機関から法令に基づき開示を求められた場合

        会員及びユーザーによる本サービスの利用状況の集計及び分析を行い、これを新規サービスの開発等の業務の遂行のために利用する場合

        会員、当社又は当社への情報提供者の正当な利益を保護するために必要な場合

第23条(個人情報の取扱い)

    本サービスの提供に際して知り得た会員及びユーザーの個人情報について、当社は、ホームページ上に記載する「プライバシーポリシー」に則り、適正に取り扱うものとします。

    当社は、会員及びユーザーの認証を行うために、システムに登録されたID及びパスワードの情報又はIPアドレスの情報を使用するものとします。

第24条(会員資格の抹消等)

    次の各号のいずれかの事由に該当する場合、当社は、催告を要さずに、会員の会員資格及びユーザー資格の全部又は一部を、将来に向かって抹消し、又は、当該事由が解消されるまでの間、一時停止することができるものとします。

        会員が当社に対し虚偽の申告をした場合

        会員又はユーザーがID又はパスワードを不正に使用し、そのほか本サービスを不正に利用した場合

        会員又はユーザーが本サービスの運営を妨害した場合

        会員が支払いを停止した場合

        会員の資産・信用又は事業に重大な変化が生じ、本規約に基づく債務の履行が困難になるおそれがあると認められる場合

        会員又はユーザーが本規約に違反した場合

        会員又はユーザーが当社又は本サービスの名誉、信用を著しく毀損した場合

        その他当社が会員又はユーザーとして不適当と判断した場合

第25条(免責)

    当社は、本規約に従って本サービスを提供している限り、会員又はユーザーが本サービスを利用したことにより、又は本サービスを利用できなかったことにより被った損害について、一切の責任を負いません。

    前項を除いて当社が会員又はユーザーに対し損害賠償義務を負う場合、当社に故意又は重大な過失があるときを除き、当社の賠償額の総額は、当社がそれまでに会員から支払いを受けた本サービスの利用料等の総額を上限とするものとします。

第26条(合意管轄裁判所)

    本規約に関する一切の紛争については、第一審の専属的合意管轄裁判所を東京地方裁判所とします。

第27条(準拠法)

    本規約には、日本法が適用されるものとします。

\section{Zeta社}
第1条 会員資格

    会員とは、本規約を承認の上、インターネットを使って株式会社Zeta(以下、「当社」といいます。)が「Zeta」の名称で提供、運営するサービス(以下、「本サービス」といいます。)の利用のために、会員として入会を申し込み、当社が入会を認めた者のことをいいます。

    会員は、本規約に基づき本サービスを利用するものとします。

    本サービス内の各サービスにおいて別途規約(以下、「個別規約」といいます。)が定められている場合は、会員は本規約及び個別規約に基づき本サービスを利用するものとします。なお、本規約と個別規約に定める内容が異なる場合には個別規約に定める内容が優先して適用されるものとします。

    会員は会員資格を第三者に利用させ、又は貸与、譲渡、売買、質入等をすることはできないものとします。

第2条 会員規約の変更

    当社は、本規約の変更が、本規約に基づく契約の目的に反せず、かつ、変更の必要性、変更後の内容の相当性その他の変更に係る事情に照らして合理的なものであるときは、民法548条の4の規定により、本規約及び個別規約を変更することができるものとします。本規約及び個別規約を変更した場合、料金その他の本サービスに関する一切の事項は変更後の規約によるものとします。

第3条 入会

    会員になろうとする方は、本規約を承認の上、当社の定める手続きにより当社に入会を申し込むものとします。

    会員になろうとする方が未成年である場合は、本サービスを利用することについて、法定代理人の同意を得るものとします。

    日本国外に在住の方は入会できません。

    18歳未満の方は入会できません。

第4条 通信端末及びID、パスワード

    会員は、当社が付与する認証用データを記録した携帯電話端末等の通信端末(以下、「通信端末」といい、当該通信端末が通信を行うためにSIMカード等のICカード等が必要な場合、当該ICカード等も含みます。)並びにID、パスワードの管理責任を負うものとします。

    会員は、会員資格を有する間、ID及びパスワードを第三者に利用させ、又は、貸与、譲渡、売買、質入等をすることはできないものとします。また、会員は、通信端末を他者に貸与、譲渡、売買、質入等する場合、会員資格が他者に利用等されないよう適切な措置を施すものとします。

    通信端末、ID及びパスワードの管理不十分、使用上の過誤、第三者の使用等による損害の責任は会員が負うものとし、当社は、当社の責めに帰すべき事由による場合を除き、一切責任を負いません。

    会員は、ID及びパスワードを第三者に知られた場合、通信端末を第三者に使用されるおそれのある場合には、直ちに当社にその旨連絡するとともに、当社の指示がある場合にはこれに従うものとします。

第5条 会員記述情報について

    会員記述情報とは、本サービス内にて会員が自ら送信、投稿、登録、表示(以下、これらの行為を単に「記述」といいます。)したすべての情報をいいます。会員記述情報に対しては、これを記述した会員が全責任を負うものとします。会員は以下の情報を記述することはできません。

        他人の名誉又は信用を傷つけるもの

        わいせつな表現又はヌード画像を含むもの

        詐欺的、虚偽的、欺瞞的である、若しくは誤解を招くもの

        暴力的若しくは脅迫的である、又は他者に対して暴力的若しくは脅迫的な行為を助長するもの

        特許権、実用新案権、意匠権、商標権、著作権、肖像権その他の他人の権利を侵害するもの

        公序良俗に反するもの

        法令に違反するもの又は違反する行為を助長するもの

        その他不適当なもの

    当社は、会員記述情報が本規約に違反する場合、その他の不適当な場合には、会員記述情報を削除することができるものとします。

    当社は、会員記述情報を本サービスの提供及び利用促進に必要な範囲において、無償で、当社サーバーへの複製、会員に対する公衆送信その他の方法により利用することができるものとします。ただし、会員間によりやりとりされる情報のうち、閲覧できる会員が特定された情報を、令状等による場合を除き、当社、第三者(会員が閲覧を許可した第三者を除きます。)が閲覧することはありません。なお、会員記述情報の著作権は当社に譲渡されるものではありません。

第6条 個人情報について

    会員になろうとする方は、当社所定の情報を当社に登録する必要があります。

    会員の本サービスの利用履歴は、会員が当社に届け出たニックネームとともに本サービス上で当社が定める期間、公開されます。

    当社は、会員の個人情報を以下の目的で利用することができるものとします。

        ゲーム、オークション、ショッピングモール、コンテンツその他の情報提供サービス、システム利用サービスの提供のため

        当社及び第三者の商品等(旅行、保険その他の金融商品を含む。以下同じ。)の販売、販売の勧誘、発送、サービス提供のため

        当社及び第三者の商品等の広告又は宣伝(ダイレクトメールの送付、電子メールの送信を含む。)のため

    当社は、以下に定める場合には、会員の個人情報を第三者に提供することができるものとします。

        裁判所、検察庁、警察、税務署、弁護士会又はこれらに準じた権限を有する機関から開示を求められた場合

        会員が当社に対し支払うべき料金その他の金員の決済を行うために、金融機関、クレジットカード会社、回収代行業者その他の決済又はその代行を行う

        当社が行う業務の全部又は一部を第三者に委託する場合

        当社に対して秘密保持義務を負う者に対して開示する場合

        当社の権利行使に必要な場合

        合併、営業譲渡その他の事由による事業の承継の際に、事業を承継する者に対して開示する場合

        個人情報保護法その他の法令により認められた場合

    当社は、会員に対し、第三者の広告又は宣伝等のために電子メールその他の広告宣伝物を送信できるものとし、会員はこれを予め承諾するものとします。

    会員は個人情報保護法に違反する行為を行ってはならないものとします。

第7条 その他の情報の取得及び利用について

    当社は、本サービス及び本サービスに付随するサービスの提供のために、当社が取得した情報のうち、個人情報に該当しない情報を統計情報としたものを第三者に提供できるものとします。

第8条 会員規約の違反等について

    会員が以下の各号に該当した場合、当社は、当社の定める期間、本サービスの一部若しくは全部の利用を認めないこと、又は、会員の会員資格を取り消すことができるものとします。ただし、この場合も当社が受領した料金を返還しないものとします。

        会員登録申込みの際の個人情報登録、及び会員となった後の個人情報変更において、その内容に虚偽や不正があった場合、又は重複した会員登録があった場合

        本サービスを利用せずに1年以上が経過した場合

        他の会員に不当に迷惑をかけた場合

        反社会的勢力と不適切な関係にある又はそのおそれのある場合

        本規約及び個別規約に違反した場合

        その他、会員として不適切である場合

    当社が会員資格を取り消した会員は再入会することはできません。

第9条 サービスの提供条件

    当社は、メンテナンス等のために、会員に通知することなく、本サービスを停止し、又は変更することがあります。

    本サービスを利用するために必要な機器、通信手段などは、会員の費用と責任で備えるものとします。

    当社は、本サービスに中断、中止その他の障害が生じないことを保証しません。

    当社は、当社が提供するアプリケーションを現状有姿で提供するものであり、当該アプリケーションが正常に動作すること及び当該アプリケーションに瑕疵のないことを保証しません。

    18歳未満の方は、本サービスを利用することはできません。

第10条 禁止事項

会員は、以下の行為を行ってはならないものとします。

    当社が提供するアプリケーション、当社が保有するサーバー及びこれらが生成する情報、通信内容等の解読、解析、逆コンパイル、逆アセンブル又はリバースエンジニアリング

    他の会員若しくはライブストリーミング配信者の個人情報、又は会員記述情報を違法、不適切に収集、開示その他利用すること

    他の個人又は団体になりすまし、又は他の個人又は団体と関係があるように不当に見せかけること

    他の会員又はライブストリーミング配信者のID、パスワードを入手しようとすること

    その他不適当なもの

第11条 コイン

    コインとは、当社の指定するコンテンツ(デジタルアイテムを含みます。)を使用するためのポイントをいいます

    会員は、コインを当社の定める方法により使用することで、当社の定める範囲のコンテンツの使用権を取得することができるものとします。コインは当社の指定するサービス内でのみ使用することができます。

第12条 料金

    会員は、当社の定める有料コンテンツを利用する場合には、当社の定める金額の利用料金を当社の定める方法により当社の定める時期までに支払うものとします。

    会員が当社の定める期日までに当社の定める利用料金を支払わなかった場合、会員は、当社に対し、支払期日の翌日より年14.6パーセントの割合による遅延損害金を支払うものとします。

第13条 コンテンツ使用許諾の条 件

    会員は、本サービスのコンテンツ(ライブストリーミング配信、アプリケーション、ウェブページ、デジタルアイテムその他本サービスにおいて提供される情報等)を、電気通信回線を通じて当社の指定する設備に接続し、通信端末に表示又はダウンロード等することによって当社の定める範囲内でのみ使用することができるものとします。

    本サービス内で当社が提供する全てのコンテンツに関する権利は当社又は当社にコンテンツの配信を許諾若しくはコンテンツの配信を委託した権利者に帰属するものとし、

    会員に対し、当社が有する特許権、実用新案権、意匠権、商標権、著作権、ノウハウその他の知的財産権の実施又は使用許諾をするものではありません。

    会員は、本サービスにおいて配信されるコンテンツを複製(私的使用のための複製を除く)、翻案、公衆送信、その他の方法により利用してはならないものとします。

    会員は、本サービスのコンテンツにつき再使用許諾をすることはできないものとします。

    本サービスのコンテンツの使用許諾は、非独占的なものとします。

    当社は、各コンテンツの使用権の有効期間を変更することができるものとします。

    退会等により会員が会員資格を喪失した場合は、コンテンツの使用権も消滅するものとします。

第14条 当社の責任

    当社は、本サービスの内容、ならびに会員が本サービスを通じて入手したコンテンツ及び情報等について、その完全性、正確性、確実性、有用性等につき、いかなる責任も負わないものとします。

    会員は自らの責任に基づいて本サービスを利用するものとし、当社は本サービスにおける他の会員及びライブストリーミング配信者の一切の事項について何らの責任を負いません。

    当社は、ライブストリーミング配信者が配信・記述する一切の事項について何ら責任を負いません。

    会員は法律の範囲内で本サービスをご利用ください。本サービスの利用に関連して会員が日本及び外国の法律に触れた場合でも、当社は一切責任を負いません。

    本規約において当社の責任について規定していない場合で、当社の責めに帰すべき事由により会員に損害が生じた場合、当社は、1万円を上限として賠償するものとします。

    当社は、当社の故意又は重大な過失により会員に損害を与えた場合には、その損害を賠償します。

    当社は、本サービスに関して、会員同士、ライブストリーミング配信者若しくはその他の第三者との間で発生した一切のトラブルについて、関知しません。したがって、これらのトラブルについては、当事者間で話し合い、訴訟などにより解決するものとします。

第15条 登録事項の変更

    会員は、メールアドレス等の登録事項に変更のあった場合、すみやかに当社の定める手続きにより当社に届け出るものとします。この届出のない場合、当社は、登録事項の変更のないものとして取り扱うことができるものとします。

    会員は、登録事項を変更したことを当社に届け出なかった場合、本サービスを利用できなくなることがあります。

第16条 当社からの通知

    当社からの通知は、当社に登録されたメールアドレスにメールを送信すること又は当社が提供するアプリケーションの機能を用いた通知方法をもって行い、メール又はアプリケーションによる通知が通常到達すべきときに到達したものとします。

第17条 サービス廃止

    当社は当社の都合によりいつでも本サービスを廃止できるものとします。

第18条 退会

    会員は、当社の定める手続きにより退会することができます。

    当社は、会員が退会した場合も当社が受領した有料コンテンツの料金を返還しないものとします。

第19条 準拠法

    本サービスその他の本規約に関する準拠法は日本法とします。

第20条 管轄裁判所

    本サービスに関し、会員と当社との間で訴訟が生じた場合、東京地方裁判所を第1審の専属的合意管轄裁判所とします。

第21条 前払式支払手段

    当社が別途「資金決済法に基づく表記」と題するページに前払式支払手段として表示するポイントは、資金決済法に基づき前払式支払手段として取扱われます。当該前払式支払手段から購入されたその他のコンテンツは、取得をもってこれにかかる商品・サービスの提供がなされたものとし、前払式支払手段には該当しません。

\section{Eta社}
第1条(目的)

    イータWEB会員規約(以下「本規約」といいます)は、株式会社イータ(以下「当社」といいます)が運営するウェブサイト(以下「本サイト」といいます)により提供するサービス(以下「本サービス」といいます)をイータWEB会員が利用するための条件を定めたものです。

    本サービスは、イータWEB会員のみが利用できます。

第2条(定義)

本規約において使用される用語の定義は以下のとおりとします。

    「イータWEB会員」とは、グループID規約に基づき登録を行ったお客さまで、本規約を承諾のうえ当社が指定する手続きにより入会を申請し、当社が承諾した者をいい、以下、「会員」といいます。

    「ID」とは、本サービスを利用するために必要なIDを意味し、会員がグループID規約に基づきIDとして登録を行ったメールアドレスのことをいいます。

    「各社」とは、当社を含め、株式会社イータのグループ各社をいいます。

    「グループID規約」とは、株式会社イータが定める「イータ・グループID規約」および「イータ・グループID規約における個人情報の取扱いについて」の総称をいいます。

    本規約において定義されていない用語は、グループID規約に基づき、定義されている意味を有するものとします。

第3条(入会・退会)

    入会希望者は、グループID規約に基づきお客さまとして登録を行ったうえで、本サイトの入会登録画面において、個人情報およびその他必要な情報を入力し、本規約を承諾することにより入会を申し込むものとします。

    当社は、入会希望者が次のいずれかに該当する場合は、入会登録を承諾しません。

        過去に本規約に基づき会員資格を抹消されている場合。

        申し込み時点または過去のいずれかの時点において第6条第1項に定めるいずれかの事由が生じていた場合。

        グループID規約第3条第2項各号のいずれかに該当する場合。

        その他、当社が会員として不適切と判断した場合。

    会員は、入会後、本サイトの退会登録画面において、随時退会の手続きを行うことができるものとします。

    会員は、グループID規約に基づきお客さまとして登録が削除、抹消または無効となった場合、本サービスを利用することができなくなります。

第4条(ログインIDおよびパスワード)

    会員は、入会後、ログインIDおよびパスワードの使用、管理およびその他の取扱いについて、グループID規約の第4条(グループIDおよびパスワード)を遵守し、会員は、自己の責任においてIDおよびパスワードの管理を行うものとします。

第5条(会員の情報)

    会員は、入会時に登録するすべての情報に関して真実かつ正確な情報を申告するものとします。

    会員は、登録した情報の内容に変更が生じた場合は、すみやかに本サイトのイータWEB会員専用ページにログインのうえ、変更手続きを行うものとします。

    当社は、会員が入会時および登録情報の変更時に登録した個人情報は、グループID規約および当社が定める「イータWEB会員における個人情報の取扱いについて」に基づいて取扱うものとします。

第6条(本サービスの利用停止または会員資格の抹消)

    当社は、会員が次のいずれかに該当した場合には、会員の有する期限の利益を喪失させ、会員へ事前に通知または催告等何らの手続きを要しないで、直ちに本サービスの利用停止または会員資格の抹消を行うことができるものとします。

        入会登録をした者が会員本人ではない場合または会員資格を喪失した場合。

        虚偽の申請をしたことが判明した場合。

        グループID規約第8条(本サービスの利用停止または登録の抹消等)第1項各号に該当した場合。

        グループID規約第9条(禁止事項)、第11条(知的財産権等)または第12条(反社会的勢力の排除)のいずれかに違反したとき。

        本規約またはグループID規約に違反した場合。

        その他、当社が会員として不適格と判断した場合。

    前項に基づく本サービスの利用停止または会員資格の抹消によって会員に生じた損害については、当社は一切その責任を負わないものとします。

第7条(本サイトおよび本サービスの運営、中断・停止および内容の変更)

    当社は、本サイトおよび本サービスの運営に関して裁量権を有しており、本サイトおよび本サービスの利用を監視し、もしくは一部変更し、またはアクセス制限などの措置をとることができるものとします。

    当社は、グループID規約第10条(本サイトおよび本サービスの運営、中断・停止および内容の変更)に基づき、自らの判断により本サイトおよび本サービスの運営の全部または一部を中断または停止できるものとします。

第8条(免責事項)

    当社は、会員に提供する本サイトもしくは本サービスにおいて会員に提供する情報および本サービスの内容等について、その完全性、正確性、確実性、有用性等につき、保証しないものとします。

    当社は本サービスにおけるプライバシー保護やセキュリティ対策にSSLを用いた暗号化技術(インターネットにおける情報を暗号化する技術)を使用しておりますが、その安全性については当社によって保証されるものではありません。

    当社は、本サービスに関連して送信されるメール等にコンピュータウィルス等の有害なものが含まれていないことについて、保証しないものとします。

    当社は、会員が登録した情報に関して、記入漏れ、メールアドレスの誤入力、判読不能な文字化け現象等、当社の責めに帰すべき事由がない場合に会員に生じた損害について、一切の責任を負わないものとします。

    当社は、本サイトの利用に係る不具合、欠陥もしくは障害の発生、会員に提供する本サービスの全部または一部の変更、追加、中断、停止、その他当社が提供する本サービスに関連して会員に生じた損害について、本規約に明示的に定める場合を除き、一切の責任を負わないものとします。第三者が当該会員になりすましてログインした場合も同様とします。

第9条(クッキーの使用等)

会員は、本サービスを利用するにあたり、予め次の事項を承諾するものとします。

    本サイトには、会員の利便性に資するため、本サイトを利用する会員情報を収集し、記録管理する技術である「Cookie(クッキー)」を使用しているページがあります。

    クッキーを拒否するように会員のコンピュータで設定された場合には、一部の本サービスがご利用いただけません。

第10条(調査協力)

    当社は、本サイトまたは本サービスの利用状況に不審な点が見受けられると判断した場合等に、会員に対し調査の協力を要請することができるものとし、会員は合理的な範囲でこれに協力するものとします。

第11条(準拠法)

    本規約の成立、効力、履行および解釈に関しては、日本法が適用されるものとします。

第12条(協議および合意管轄)

    本規約に関連して、会員と各社の間で争いが生じた場合は、会員と各社との間で誠意をもって協議し、解決するものとします。

    前項に基づき協議したにもかかわらず、会員と各社との間の紛争が解決できなかった場合は、東京地方裁判所を第一審の専属的合意管轄裁判所とします。

第13条(規約の変更等)

    当社は、会員の事前の承諾なしに、本条に従い任意に本規約を変更できるものとします。

    当社は、本規約の変更をしたときには、一定の予告期間を設けて(但し、会員の一般の利益に適合する場合を除きます)、本サイトへの公開その他当社が適当と判断する方法により当該変更後の本規約を会員に通知または公表するものとします。

    変更後の本規約は、前項により当社から会員に通知または公表した後、当該予告期間が経過した時点から効力が生じるものとします。但し、会員の一般の利益に適合する場合には、当該変更は、当社から会員に対する通知または公表が行われたと同時にその効力が生じるものとします。

    会員は、本規約変更後の本サイトまたは本サービスの最初の利用をもって、当該変更に同意したものとみなします。但し、商品購入申し込みの手続き後に本規約が変更された場合は、会員が当該申し込みを行った時点での本規約(前項によって判定されるものとします)が有効となるものとします。

第14条(適用範囲)

    本章は、会員が本サイト上で商品の購入を行う場合に適用されます。

    本サイト上での商品購入の取引は、会員と各社との間で直接行われます。

第15条(契約の成立・解除)

    各社は、以下に定める事由のいずれかに該当する場合には、商品売買契約が成立した後でも、会員に対する解除通知を行ったうえで、商品売買契約の一部または全部を解除することができるものとします。この場合、各社は会員に対して損害賠償義務を負わないものとします。

        会員が各社の定める期限までに商品代金を支払わなかった場合。(クレジットカード決済の承認が得られない場合を含みます。)

        会員が本規約に違反した場合または本サービスに関連して当社が定める規定等に違反した場合。

第16条(申し込み手続き等)

    会員が、本サイトにより、商品購入をするときの申し込み手続き、支払い方法、お届け方法、未着等に関するお問い合わせおよび購入した商品の取り替え・返品等の具体的な事項については、当社が別途掲示する「ご利用ガイド」によるものとします。

第17条(ショッピングカートからの商品削除)

    各社は次のいずれかに該当した場合には、会員へ事前に通知することなく、商品売買契約の成立の前に、会員が本サイトのショッピングカートに入れた状態にある商品のすべてを削除することができるものとします。

        会員がショッピングカートに商品を入れてからご注文の確定をせず一定の時間を経過した場合。

        会員が不必要に大量の商品をショッピングカートに入れ、他の会員の商品購入の機会を阻害するなど、他の会員、第三者、各社に不利益もしくは損害を与える行為、または、与えるおそれがあると各社が判断したとき。

第18条(免責事項)

    会員が登録情報の変更を怠った場合、会員が登録した情報に関して、事実と異なる記入、記入漏れ、メールアドレスの誤入力、判読不能な文字化け現象が生じた場合、第7条第2項に定める事由が生じた場合、各社が申し込み時に指定されたお届け先に納品に赴いたにもかかわらず宛先不明、長期不在(再配達の手続きがとられない場合を含みます)等で引渡しが完了できない場合、店頭受け取り指定の場合に、各社が指定した期限までに申し込み時に指定された店舗へ会員が来店せず引渡しが完了できない場合、その他各社の責めに帰すべき事由なくして商品の提供ができない場合、各社は一切の責任を負わないものとします。

第19条(定義)

    コミュニティサービス(以下「本コミュニティサービス」といいます)とは、本サイトにより提供する、会員がコメント、レビュー、写真等を投稿できる機能を有するサービス、ならびに当社または各社および会員同士で交流できる機能を有するソーシャル・ネットワークサービス全般をいいます。
    
第20条(適用範囲)

    本章は、会員が本コミュニティサービスを利用する場合に、第1章総則の規定とともに適用されるものとします。

第21条(情報の取扱い)

    会員は、自らが本コミュニティサービスに掲載、投稿した内容について一切の責任を負うものとします。

    当社は、会員から本コミュニティサービスに掲載、投稿された内容について、会員の承諾なしに次に掲げる事項を行うことができるものとします。

        内容について審査を行うこと。

        本コミュニティサービスに掲載すること、または掲載しないこと。

        本コミュニティサービス以外で、当社が認めた媒体に掲載すること、または掲載しないこと。

        当社が必要と判断したときは、会員に対し、連絡を行うこと。

第22条(禁止事項)

    会員は、本コミュニティサービスの利用に際し、原則としてグループID規約第9条に従うほか、次に掲げる行為、もしくはそのおそれのある行為を行ってはならないものとします。

        他の会員、その他第三者または当社を誹謗中傷し、侮辱し、名誉、信用、プライバシーを毀損、または当社の業務を妨害する行為。

        他の会員、その他第三者または当社に関係する個人を特定し得る情報(メールアドレス、電話番号、住所等)を開示、または提供する行為。

        他の会員、その他第三者、または当社の権利・利益を侵害する行為。

        同一内容および同一と思われる内容を多数投稿、または当該投稿を繰り返す行為。

        広告・宣伝その他営業活動等、営利目的に利用する行為。

        当社が不適切と判断する他のウェブサイトを紹介し、または閲覧を勧誘する行為。

        公職選挙法に違反する行為。

        特定の思想および宗教団体、その他の団体・組織への加入等を勧誘する行為。

        寄付、出資、資金提供、物品またはサービスの購入等を勧誘する行為。

        事実と反する、または虚偽の内容を含む情報を提供する行為。

        児童、青少年の健全な育成に悪影響をおよぼす内容を含む情報を提供する行為。

        本コミュニティサービスを出会い系の目的で利用する行為。

        その他当社が不適切と判断する行為。

        前項に該当する場合のほか、当社の管理運営上の都合その他特に理由を必要とせず、当社の任意の判断で投稿された情報を会員に通知することなく削除でき、会員はこれに異議を述べることはできないものとします。

第23条(投稿情報の利用)

    当社は、会員が本コミュニティサービスを通じて掲載、投稿した内容について、当該会員に通知することなく、自由に転載、引用、開示、提供、出版、配信その他の方法により、無償で利用できるものとし、会員は、当社および当社の指定する者に対し、著作権(著作権法第27条および第28条に規定される権利を含む)および著作者人格権を行使しないものとします。

    会員は、当社の事前の書面による承諾を得ることなく、他の会員による掲載、投稿内容を含む本コミュニティサービスの内容を転載、転用、編集、複製その他一切の利用行為を行ってはならないものとします。

第24条(損害賠償)

    会員は、本コミュニティサービスの利用に関連して、会員と他の会員または会員とその他第三者との間で争いが生じた場合は、当該会員は自己の責任と負担においてその解決を図り、その賠償をし、当社に対し一切の負担および迷惑をかけないものとします。但し、当社の責めに帰すべき事由がある場合は、この限りではありません。

    前項但し書きにより当社が責任を負う場合であっても、当社の故意または重大な過失がない限り、当社の責任は、直接かつ通常の損害に限られるものとします。

\section{Theta社}
第1条 適用

    本規約は、ユーザーと当社との間の本サービスの利用に関わる一切の関係に適用されるものとします。

    ユーザーは、本規約に同意する場合のみ本サービスを利用できるものとします。

第2条 ユーザーアカウント

    ユーザーは、当社が定める方法によって、本サービスの利用に必要なユーザーIDおよびパスワードを付与され、ユーザーアカウントを取得するものとします。ユーザーは、ユーザーアカウントの取得後に、当社が定める手順に従って必要なユーザー情報の登録を行うものとします。ユーザー登録の入力内容については、ユーザー自らが責任を負うものとします。

    当社と、ユーザーが所属する学校・企業・地方公共団体等との間で、ユーザーの本サービスの利用に関する契約が締結された場合には、当社から学校・企業・地方公共団体等に対してユーザーアカウントが付与され、学校・企業・地方公共団体等からユーザーアカウントがユーザーに提供されます。

    当社が発行したユーザーアカウントは、付与されたユーザーのみが利用可能なものとし、第三者への譲渡、貸与を禁止します。

    ユーザーは、ユーザーIDやパスワードの取り扱い等、自らの責任のもとでユーザーアカウントの管理を厳重に行うものとします。発行されたユーザーアカウントの利用による本サービスでの行為は、すべて付与されたユーザーアカウントのユーザーのものとみなします。

    ユーザーが当社の定めた期間以上に、本サービスへのログイン等の利用がみられない場合、または本規約第10条に違反したユーザーなど、本規約の規定に従って当社がユーザーアカウントの停止および契約の自動更新の停止を行うことが必要と認めた場合には、ユーザーに対し事前に通知をした上で、ユーザーアカウントの停止および契約の自動更新の停止を行うことができるものとします。

第3条 ユーザーID・パスワードの管理

    ユーザーは、本サービスを利用するためのユーザーID(株式会社シータが提供する会員用ページにログインするためのIDなど、本サービス利用に用いるさまざまな種類のID類を含みます)およびそれぞれのパスワードを自己の責任で使用し、第三者に使用させないように管理するものとします。当社の責任に拠らずに、これらを第三者が使用したことにより生じた事態や損害については、当社は責任を負わないものとします。

第4条 本サービスのアカウント登録、有料契約

    本サービスのユーザーは、本規約に同意の上、本サービスへアカウント登録を行うことで契約が成立し、本サービスを利用できるようになります。

    本サービスのアカウント登録は、本サービスの「アカウント登録サイト」および当社ウェブサイトのお申し込みフォームにて承りますが、その他の方法であっても当社が異議なく受諾した登録については、本規約に従う登録として取り扱うものとします。

    本サービスのうち、当社が有料での利用を規定するサービス(以下、「有料サービス」)について、利用契約を交わしたユーザーを契約者とよびます。契約者が、本サービス内の有料サービスの利用契約を希望する場合には、本規約の内容、有料サービスの契約内容を承諾の上、当社の定める手順に従い利用契約を取り交わすものとします。

    本サービスのユーザーが未成年の場合は、当該ユーザーの保護者の方、ユーザーが18歳、19歳の場合は、当該ユーザーの保護者だった方(以下、保護者の方と保護者だった方を併せて「保護者等」といいます)に事前に同意を得た上で、本サービスの利用・利用契約を行うものとします(契約者は、ユーザーが未成年の場合は、当該ユーザーの保護者の方、ユーザーが18歳、19歳の場合は、当該ユーザーと当該ユーザーの保護者だった方の両者となります)。

    保護者等は、ユーザーが本サービスで行う一切の行為および契約について、いかなる場合においても連帯して責任を負うものとします。

第5条 支払いについて

    本サービスの支払い方法はクレジットカード決済、コンビニ決済のいずれかとなります。なお、支払いに要する手数料などはユーザーにご負担いただきます。

    本サービス内の有料サービスの利用料金は、別途定めるものとします。

    当社は、ユーザーの承諾なく、利用料金等を変更することができるものとします。その場合、当社は、事前にユーザーに告知するものとします。

    ユーザーから当社への利用料金の支払いは、第三者を通じて行われるため、利用料金の支払いに関する領収書を当社が発行することはありません。

    ユーザーが、クレジットカード会社の定める規約の事由に該当し、または別の理由により、決済方法の利用を停止された場合には、当社は何らの予告・通知なく、ユーザーアカウントの停止を行うことができるものとし、未払の利用料がある場合は別の方法による支払を求めることがあります。ユーザーはそれに異議を述べないものとします。

    ユーザーがクレジットカード決済、コンビニ決済を選択した場合、当社は請求業務を委託できるものとします。

    当社が委託契約等を締結する適格業務代行会社は認証を得ています。

第6条 解約について

    申込手続き完了後の解約はできません。

第7条 必要となる機器等

    本サービスを利用するために必要な、パーソナルコンピュータ、タブレット端末、通信機器、ソフトウェア、インターネット接続環境など、当社が設置・管理・運用に関与しない一切の機器およびソフトウェア等については、ユーザーが自己の費用と責任で準備し、維持するものとします。また、本サービスの利用に関わる通信料等はユーザーの負担となります。

第8条 著作物の取り扱い

    本サービスに関連して提供される情報その他の著作物(以下「著作物等」という)に関する権利は、当社または当該著作物等の著作権者に帰属します。ユーザーは、当社の事前の書面による承諾を得ずに、著作物等を複製、公衆送信、頒布、翻案、翻訳、及び二次的著作物への利用等を行うことはできません。

    ユーザーが本サービス内において蓄積したデータ等の権利は、一切当社に帰属するものとします。

第9条 同意事項

    ユーザーは、本サービスを利用するにあたり、以下の各項について同意するものとします。ただし、当社に故意または重大な過失がある場合には適用しないものとします。

    本サービスの利便性を向上させるために、ユーザーによる本サービスへの接続時に当社がCookieを利用すること

    当社が自動的にユーザーのアクセスログ(アクセス日時、閲覧したページIPアドレス等)を取得することおよび、取得したアクセスログと当社が保有するユーザーの個人情報とを組み合わせて、本サービスの円滑な運営のために利用すること

    本サービスの利用においてユーザーが入手する情報の安全性・確実性・信憑性・有効性その他、情報の評価に関し、当社が一切の責任を負わないこと。但し、当社に故意または重大な過失がある場合には適用しないものとします。

    ユーザーの自己責任において、コンピュータ・ウイルスの感染予防の措置を講じるものとし、当社がウイルス感染を回避する保証を与えるものではないこと

    本サービスにおいて、ユーザーごとに蓄積されたデータは、当社の定める一定の期間の後に消去され、本サービス内で閲覧できなくなること。その期間については当社の定めるものとし、ユーザーへの告知は特段の場合を除き行わないこと

第10条 禁止事項

ユーザーは、本サービスの利用について以下の各項に挙げる行為をしてはなりません。いずれかに該当する行為を行ったユーザーに対し、当社は何らの予告・通知もなく、ユーザーアカウントの停止もしくは契約の自動更新の停止を行うことができるものとします。その場合、違反行為をしたユーザーは利用を差し止められたことについて、何らの異議を唱えないものとし、また何らの補償・賠償・補填・代償・支払済利用料金の返却も求めることはできないものとします。

    他のユーザーのユーザーID、パスワードを不正に使用する行為

    1つのアカウントを複数人で使用する行為

    本サービス上の情報等を無断で改ざん・消去等する行為

    当社および第三者が権利を有する著作権などの知的財産権を侵害する行為

    意図的に虚偽の情報または誤解を招く情報を登録する行為

    当社及び第三者の通信の秘密またはプライバシーを侵害する行為

    当社及び第三者を誹謗、中傷、差別する行為

    当社が本サービス上で提供する各種サービスを不正の目的をもって利用する行為

    本サービス上で発生した不具合を利用する行為

    当社が許可しない勧誘・宣伝・販売などのユーザー本人または第三者の営利を目的とする行為

    本サービスならびに当社が運営・管理するホームページ、データベース等への不正接続行為

    本サービスを利用してコンピュータ・ウィルス、わいせつ図画などの有害データを掲示・配布・送信する行為

    わいせつ、残虐、暴力、犯罪、中傷、その他の公序良俗に反するURLへリンクする行為

    法令に違反または違反する恐れのある一切の行為

    当社および第三者の権利および利益を侵害する行為

    本サービス運営に支障をきたす行為、または当社及び第三者の権利・利益を害する行為と、当社が判断する一切の行為

    その他、本規約に定める義務に反する行為、または本サービスを妨害する一切の行為

第11条 サービスの一時停止

    当社は、本サービス用設備の保守または工事の都合、および当社が委託する電気通信事業者またはインターネットサービスプロバイダ等の都合により、本サービスの全部または一部の提供を一時的に停止することがあります。この場合、緊急を要する場合を除き、本サービスの利用画面上に表示するなど、当社が適当と判断する方法によってユーザーに通知するものとします。なお、サービスの一時停止に関しては、本サービスの利用料金の返却は行いません。

第12条 サービスの廃止・変更

    当社の都合により、本サービスの全部または一部を廃止または変更することがあります。その場合、ユーザーに対し、第11条と同様の方法により、事前に通知するものとします。

    廃止または変更によって損害が生じた場合でも、当社は一切その責を負わないものとします。ただし、当社に故意または重大な過失がある場合には適用しないものとします。

第13条 守秘義務

    当社は、当社が定める「個人情報保護方針、個人情報の取り扱いについて」に基づいて、ユーザーの個人情報を取り扱います。本サービス利用中はもとより本サービス利用終了後においても、必要な限度を超えて利用せず、第三者に個人情報を提供することもありません。

    当社は利用契約の履行に際し知り得た契約者の情報を正当な理由なく第三者に漏らしません。 

    当社は、次の場合を除き利用者のアカウントにログインし、その内部情報を入手することはないものとします。

        当社が、復旧または保守作業上必要と認めた場合

        ユーザーの依頼による場合、もしくは事前に許可を得た場合

    当社は、前項で規定する場合においてユーザーのアカウントにログインした場合、それによって知り得た内部情報等の情報を次に該当する場合を除いて第三者に漏洩しないものとします。

        ご利用者の依頼、または承諾による場合

        捜査協力等の義務が生じた場合 

第14条 遅延損害金

    ユーザーが、本規約に定める利用料金その他の債務を、支払期日までにお支払いにならなかった場合、サービスを停止するとともに、支払期日の翌日から支払日までの遅延損害金として、年率14.5%の割合で加算した金額を申し受けます。 

第15条 料金の返還

    利用料金等は、いかなる理由があっても返却しません。 

    当社の責においてサービスが停止した場合停止した事を当社が認知してから24時間以上に渡って連続してサービスが停止した場合、月額費用÷月日数÷24×停止時間の金額を返還いたします。

第16条 免責

    ユーザーは、以下の不具合等から生じた損害について、当社が一切その責を負わないことに同意するものとします。ただし、当社に故意または重大な過失がある場合には適用しないものとします。

    本サービスについて、動作確認環境以外の環境からの利用の動作保証を行わないこと

    ユーザーが使用するパーソナルコンピュータ、タブレット端末、通信機器、ソフトウェア、インターネット接続環境等によって、提供サービスや使用する機器に生じる不具合

    天災、事変、その他の不可抗力や第三者の故意など、当社の責任によらない原因によって、本サービスに障害が生じたり、損害を受けたりすること

    当社の責任によらない原因によって、本サービスで管理または提供されるデータが改ざん・消去されること

    ユーザーによる本サービスの利用にともなって生じるユーザー間の係争については、当社は、何らの責を負わないこと。

第17条 広告主との取引

    本サービスの画面上に掲載されている広告やリンク、ならびにユーザーによるPR情報について、広告主やリンク先企業、ならびにPR情報を掲載したユーザーとの連絡および取引は、取引当事者間の責任において行うものとし、当社は何ら責任を負わないものとします。

第18条 サービス、価格、仕様などの変更

    カタログ、チラシなどの有効期間中において、ユーザーに事前に通知することなく、本サービスの一部もしくは全部を変更することがあります。また、ユーザーに事前に通知することなく、掲載商品の価格または仕様が変更もしくは取扱が中止される場合があります。

第19条 準拠法・管轄

    本サービスに関して、本規約に定めの無い事柄については、ユーザーの住所・ネットワーク接続地にかかわらず、日本国の法令の定めによります。

    本サービスに関する訴訟に関しては、当社本店所在地を管轄する裁判所を第一審の専属的合意管轄裁判所とします。

第20条 利用規約の適用および変更

    当社が本サービスにおいて提示する規定、注意事項等は、本規約の一部を構成するものとします。

    当社は、当社が必要と判断する場合、いつでも本規約を変更できるものとします。変更後の本規約は、当社から変更内容とともにユーザーに通知しますが、運営するウェブサイト内に掲示された時点から効力を生じるものとします。

    ユーザーは本規約の変更後も本規約に同意した上で、本サービスを利用するものとし、本規約変更後に、本サービスを利用した場合、本規約に同意したものとみなします。

第21条 プライバシーポリシー(個人情報等の取り扱いについて)

    当社は本サービスにおいて、取得したユーザーの情報に関し、別途定める「個人情報保護方針、個人情報の取り扱いについて」に基づき、適切に取り扱うものとします。

    当社は、当社の定めた「個人情報保護方針、個人情報の取り扱いについて」に従い、ユーザー本人の在学校等のユーザーの個人情報の一部を、ユーザーの同意を確認した上で、本サービス上に表示することがあります。

第22条 協議事項

    本規約に定めのない事項および本規約の各条項の解釈について疑義が生じた事項について、ユーザーと当社は誠意を持って協議し解決するものとします。
