Abstract of Masters's Thesis - Academic Year 20xx
\begin{center}
\begin{large}
\begin{tabular}{|p{0.97\linewidth}|}
    \hline
      \etitle \\
    \hline
\end{tabular}
\end{large}
\end{center}

~ \\
The number of Internet users is increasing every year, and the number of services offered on the Internet is also increasing accordingly. In most cases, it is necessary to agree to the Terms of Service before using a service offered on the Internet. However, there is a problem that the Terms of Service are not read very often, which can cause problems. In this study, as a method to reduce the time required to read the Terms of Service and to discover problematic clauses, we propose a method to record the Terms of Service that have been agreed to, so that when reading new Terms of Service, we can extract clauses that have the same meaning as those we have read before and focus on the rest of them. This can provide assistance in reading the Terms of Service in a form that is tailored to each individual and create a more user-friendly Internet environment.
~ \\
Keywords : \\
\underline{1. Terms of Service},
\underline{2. Legal Document},
\underline{3. NLP},
\underline{4. Reading Support}
\begin{flushright}
\edept \\
\eauthor
\end{flushright}
