修士論文要旨 - 2022年度 (令和4年度)
\begin{center}
\begin{large}
\begin{tabular}{|M{0.97\linewidth}|}
    \hline
      \title \\
    \hline
\end{tabular}
\end{large}
\end{center}

~ \\

インターネット利用者は年々増加しており、それに伴いインターネット上で提供されるサービスも増加している。インターネット上で提供されるサービスを利用するためにはほとんどの場合利用開始前に利用規約への同意をする必要がある。しかし、利用規約はあまり読まれていない問題があり、そのためにトラブルが発生する可能性がある。本研究では、利用規約を読むために必要する時間を減らし、かつ、問題のある条項を発見するための手法として、同意した利用規約を記録していき、新たに利用規約を読む時に読んだことのある条文と同じ意味の文を抽出してそれ以外を注目するようにする手法を提案する。これにより、個人個人に合わせた形での利用規約の読解支援を提供でき、より利用しやすいインターネット環境を作り出すことができる。

~ \\
キーワード:\\
\underline{1. 利用規約},
\underline{2. 法的文書},
\underline{3. 自然言語処理},
\underline{4. 読解支援}
\begin{flushright}
\dept \\
\author
\end{flushright}
