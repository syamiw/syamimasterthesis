\chapter{関連研究}
\label{related}
%他人の先行研究・事例で、自分の研究が戦う相手について述べて比較する

本章では、本研究の関連研究を示す。

\section{サービス利用規約の読解促進を目指した表示手法の比較検討(竹ノ内ら 2020)}
竹ノ内ら\cite{竹ノ内2020}は利用規約の理解促進を目的として、表示手法の比較を行っている。

\section{利用規約及びプライバシーポリシーのデザインと理解度の評価(土屋ら 2022)}
土屋らは\cite{土屋2022}は利用者がサービスを利用する際のメリットやリスクなど重要なな項目を理解できるようにするため、利用規約などのわかりやすいデザインを提案している。表示項目としては、個人情報保護法に基づいて、各項目をWho(誰が個人情報を利用するか)、What(何のデータを利用するか)、Why(何の目的で利用するか)、When(いつ取得・提供するか)、Where(海外での利用はあるか)、How(どのように利用するか、問い合わせについて)の5W1H