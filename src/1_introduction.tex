\chapter{序論}
\label{introduction}
%研究の動機 (とこの研究による世界への貢献)
%この研究は大事なんだ!ということを伝える

本章では本研究の背景、課題及び手法を提示し、本研究の概要を示す。

\section{はじめに}
\label{introduction:background}
インターネットの利用者は年々増加しており、それに伴い、インターネット上で提供されるサービスも増えている。利用できるサービスが増加すると便利な機能や新しい体験をすることができるが、新しくサービスを利用したり、アカウントを作成するためにはほとんどの場合、利用規約を読み、同意することが求められる。しかし、サービスを利用するのを優先するためにその前に表示される利用規約は読まれないような状況に陥ってしまっている。だが、利用規約は当然に読まれていることが前提としてサービスを利用しているため、この状態は危険であると考えられる。

\section{本研究の定義}
本研究における「利用規約」とは、一般的にインターネット上で特定のサービスを利用する際に同意を求められる事業者がサービスの利用に関する規則を記載したものであり、他国の法律まで範囲を拡張することは困難であるため、日本の法律の下で提供される日本語の利用規約を対象とする。

\section{本論文の構成}
本論文における以降の構成は次の通りである。

~\ref{background}章では、背景を述べる。
~\ref{issue}章では、本研究における問題の定義と、解決するための要件の整理を行う。
~\ref{proposed}章では、本研究の仮説を述べる。
~\ref{experiment}章では、実験について述べる.
~\ref{discussion}章では、\ref{experiment}章で行った実験に関しての考察を行う。
~\ref{related}章では、本研究と関連研究との比較を行う。
~\ref{conclusion}章では、本研究のまとめと今後の課題についてまとめる。


%%% Local Variables:
%%% mode: japanese-latex
%%% TeX-master: "../thesis"
%%% End:
