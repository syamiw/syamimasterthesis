\chapter{序論}
\label{introduction}
%研究の動機 (とこの研究による世界への貢献)
%この研究は大事なんだ!ということを伝える

本章では本研究の背景、課題及び手法を提示し、本研究の概要を示す。

\section{はじめに}
\label{introduction:background}
インターネットの利用者は年々増加しており、それに伴い、インターネット上で提供されるサービスも増えている。利用できるサービスが増加すると便利な機能や新しい体験をすることができるが、新しくサービスを利用したり、アカウントを作成するためにはほとんどの場合、利用規約を読み、同意することが求められる。しかし、サービスを利用するのを優先するためにその前に表示される利用規約は読まれないような状況に陥ってしまっている。将来的に同意したサービスを利用した時に問題が発生した場合、ほとんど読んでいない利用規約をもとに裁判を行う必要が生じてしまうため、このときに自分が納得することができない条項などが含まれていても手遅れになってしまい、当然に同意した事実の方が優越してしまう。これにより、利用規約を読むことは重要といえるが、それを利用者に求めた場合、新しいサービスを利用するたびに利用規約を読む時間や労力を求めることとなってしまう。よって、本研究では、より安心してインターネット上のサービスを利用するために、利用規約の読解支援を行う。

\section{本研究での定義}
本研究における「利用規約」とは、インターネット上で特定のサービスを利用する際に同意を求められる事業者がサービスの利用に関する規則を記載した定型約款である。なお、定型約款については\ref{background}章で詳しく述べる。本研究では他国の利用規約まで対象にした場合は、他国で制定された法律についても取り扱う必要がありまた、利用規約の法的立ち位置は国によって異なるため取り扱うことが困難である。よって、本研究は日本の法律の下で提供されるサービスの日本語での利用規約を対象とする。

%\section{本論文の構成}
%本論文における以降の構成は次の通りである。

%~\ref{background}章では、本研究で取り扱う利用規約の法的意義と、本研究の提案手法で取り扱う自然言語処理についての概説を述べる。
%~\ref{issue}章では、利用規約を読む際の問題に点について整理を行う。
%~\ref{proposed}章では、本研究の提案手法を述べる。
%~\ref{experiment}章では、\ref{proposed}章で述べた手法の検証のためのシステムの実装方法を述べる.
%~\ref{discussion}章では、\ref{experiment}章で行った実験に関しての考察を行う。
%~\ref{related}章では、本研究の評価と本研究と関連研究との比較を行う。
%~\ref{conclusion}章では、本研究のまとめと今後の課題についてまとめる。


%%% Local Variables:
%%% mode: japanese-latex
%%% TeX-master: "../thesis"
%%% End:
